\documentclass{article}
% \usepackage[scheme = chinese]{ctex}
\usepackage[scheme = plain]{ctex}

\usepackage{indentfirst}
\usepackage{mdframed}
\usepackage{minted} % 打印源代码
\newcommand{\mnt}[1]{\mintinline{latex}{#1}} % 简化打印源代码命令
%* 上述命令存在限制,见文末注释

% 命令介绍架构
\newcommand{\notes}[5]{
    \begin{mdframed}
        \paragraph{命令}#1
        \paragraph{宏包}#2
        \paragraph{位置}#3
        \paragraph{参数}#4
        \paragraph{注释}#5
    \end{mdframed}
}

\title{\LaTeX{}学习笔记}
\author{殷元昊}
\date{}

\begin{document}
\maketitle

\section{前言}
这份笔记的前身来自笔者学习吴康隆的《简单高效LATEX》的笔记。笔者按照本书的结构学习了大部分代码,在学习的同时进行了大量的随文注释。但是在回看的时候,这些代码和随文注释混杂在一起,极大地影响了可读性。因此笔者深感有必要将这些随文注释整理出来,形成单独的\LaTeX{}文档。

在吴书之外,笔者还另外单独学习了一些有用的\LaTeX{}宏包,并也做了大量的随文注释,这些注释也需要整理出来。

值得一提的是,由于AI工具的兴起,代码学习的效率被大大提高,笔者在编辑本文档时,也仍然在使用AI学习新的知识,并立刻将其应用到本文档中

笔者曾经在学习制作表格和插入图片的相关宏包时制作过两份\LaTeX{}文档,但是回看的时候仍然觉得可读性不强,这是因为在编写文档的时候,尽管笔者使用了详细的文字描写相关命令,但由于缺少命令的参数结构,因而在回看时仍不免陷入“迷失在文字的海洋里”的尴尬境地,因此,该笔记将严格按照“宏包-位置-参数-注释”的架构搭建。

\section{正文}
\subsection{封面}
\notes
{
    \mnt{\title{}、\author{}、\date{}、\thanks{}}
}%
{基础命令}%
{导言区}%
{略}%
{
    \begin{enumerate}
        \item \mnt{\date{}}如果省略,仍然会自动打印出编译当天的日期,如果不想显示日期,可以保留\mnt{\date{}},但不输入参数。
        %todo 但是还需要补充一个知识:如何通过该命令控制输出日期的格式
        \item 标题页的脚注用\mnt{\thamks{}}完成。
    \end{enumerate}
}%

\subsection{标题}
\notes
{
    \mnt{\maketitle}
}%
{基础命令}%
{略}%
{略}%
{略}%

\subsection{目录}

\notes
{
    \mnt{\tableofcontents、\listoffigures、\listoftables}
}%
{基础命令}%
{略}%
{略}%
{略}%

\notes
{
    \mnt{\contentsname、\listfigurename、\listtablename}
}%
{基础命令}%
{略}%
{略}%
{
    这三条命令用于\mnt{\renewcommand{}{}}结构中,通过重定义来修改目录的标题、图片目录的标题、表格目录的标题。
}%

\subsection{保留字符}

\notes
{
    \mnt{\#、\$、\%、\&、\_、\{、\}、\^{}}
}%
{基础命令}%
{略}%
{略}%
{
    关于\%符号,吴书中提到:“如果在行末添加\%这个命令,可以防止LaTeX在行末插入一些奇怪的空白符”,其实这个说法语焉不详,我们知道,如果在LaTeX的代码中换一行,打印时就会在换行处插入一个空格,而此时如果在每一行的末尾插入一个\%符号,就能移除这个空格。

    关于\^{}符号,该命令后面如果不加一对大括号,单独打印时会报错,而如果在大括号中填入一个字母(实际可以只填入一个字母而不需要大括号),输出的就是一个扬抑符(circumflex),比如\mnt{\^{a}}输出\^{a}。
}%

\notes
{
    (1) \mnt{\textbackslash}、\\
    (2) \mnt{\textrm{\char92}、{\rmfamily\char92}}、\\
    (3) \mnt{$backslash$}
}%
{基础命令}%
{略}%
{略}%
{
    以上是输出反斜杠的三种方法,第二种方法使用ASCII码进行输出,第三种方法使用数学环境,\mnt{\textrm{}}也可以替换为其他字体命令。

    第二种方法、第三种方法输出的反斜杠和第一种方法输出的反斜杠不完全相同:如果将这几种方法输出的反斜杠排列在一行上,会发现第一种方式输出的反斜杠和后面反斜杠符号的间距更小,但是有意思的是,如果将反斜杠后面的符号替换成其他符号,这些反斜杠后的这一间距又会恢复相同,因此在实际情况中,只需要在需要连续使用两个反斜杠符号时,注意选择命令即可,比如,如果要输出“\textbackslash\textbackslash”符号,应该选用两个连续的\mnt{\textbackslash}命令。

    顺便在此可以讨论\mnt{\texttt{}}和\mnt{{\ttfamily}}这两类不同的命令,这两类命令的大括号的位置和功能是不同的:前一种命令的大括号置于命令之后,是强制性的,用来放置相应的参数,只有大括号内的内容才会受到该命令的影响,其特点有点类似于环境。

    而后一种命令的大括号不是强制性的,如果没有这一对大括号,则该命令之后的所有内容都会受到该命令的影响,如果需要限制该命令影响的范围,则可以在该命令和需要统辖的范围两边加上大括号。

    下文中会反复出现这两类命令,以下只通过大括号的位置来体现这两类命令的类型,如果没有必要,不再就这一点展开具体讨论。
}%

\notes
{
    (1) \mnt{\textasciitilde}、\\
    (2) \mnt{\~{}}、\\
    (3) \mnt{$sim$}
}%
{基础命令}%
{略}%
{略}%
{
    以上是输出波浪线的几种方法。
    
    方法一在不同的编辑器中输出效果不同,在TeXstudio中输出的是一个腭化符,而在VS Code中输出的却是一个正常的波浪线。

    方法二和上文中的\^{}符号一样,该命令如果后面跟上一个字母,实际输出的就是一个波浪号/腭化符(tilde),如果要单独输出这一符号,必须在后面添加一对大括号,在TeXstudio中输出的是一个单独的腭化符,但是在VS Code中输出的就是一个正常的波浪线,比如\mnt{\~{a}}输出\~{a}。

    方法三实际是一个数学符号“约等于”(similar)。
}%

\subsection{大于号和小于号}
\notes
{
    (1) \mnt{\textgreater、\textless}、\\
    (2) \mnt{>、<}、\\
    (3) \mnt{$>$、$<$}
}%
{基础命令}%
{略}%
{略}%
{
    以上是输出大于号和小于号的几种方法。

    文本中的大于号和小于号需要使用\mnt{\textgreater}和\mnt{\textless}命令。

    方法二和方法三输出的都是数学符号中的大于号和小于号。TeXstudio中直接输入大于号和小于号不会正确打印相应的符号,但是VS Code中正常,效果等同于两边加上数学环境。
}%

\subsection{引号}
\notes
{
    \mnt{“你好,‘世界’!”} v.s. \mnt{``\thinspace`Max' is here.'' }
}%
{基础命令}%
{略}%
{略}%
{
    中文下的单引号和双引号可以用中文输入法直接输入。英文的左单引号是重音符“\mnt{`}”,右单引号是常用的引号符“\mnt{'}”。

    吴书中提到上述“英文下的引号嵌套需要借助\mnt{\thinspace}命令分隔”,但实际上这和语言无关,\mnt{\thinspace}的效果是在命令处略微扩大外部引号和内部引号之间的距离,只是一个细节问题。
}%

\subsection{连字符、破折号和省略号}
\notes
{
    (1) \mnt{daughter-in-law}、\\
    (2) \mnt{1--2}、\\
    (3) \mnt{Listen---I'm serious.}、\\
    (4) \mnt{——}、\\
    (5) \mnt{……}、\\
    (6) \mnt{\ldots} 
    % ldots = lower dots, v.s. \cdots = center dots
    %todo 该命令效果也等同于\dots,但是\ldots在任何模式下都适用,关于dots的相关命令有待探索
}%
{基础命令}%
{略}%
{略}%
{
    以上命令输出的分别是(1)连字符、(2)数字起止符、(3)英文破折号、(4)中文破折号、(5)中文省略号、(6)英文省略号。注意连打三个句点\mnt{...}输出的不是真正的英文省略号。
}%

\subsection{强调}
\notes
{
    \mnt{\emph{强调} v.s. \emph{empasis}}
}%
{基础命令}%
{略}%
{略}%
{
    中文的效果等同于\mnt{\textsl{}},西文的效果等同于\mnt{\textit{}}。
}%

\subsection{下划线和删除线}
\notes
{
    \mnt{\underline{}}
}%
{基础命令}%
{略}%
{略}%
{略}%

\notes
{
    (1) \mnt{\uline{}}、\\
    (2) \mnt{\uuline{}}、\\
    (3) \mnt{\dashuline{}}、\\
    (4) \mnt{\dotuline{}}、\\
    (5) \mnt{\uwave{}}、\\
    (6) \mnt{\sout{}}、\\
    % “sout”来自“strike out”的缩写
    (7) \mnt{\xout{}}
    % “xout”来自“cross out”的缩写
}%
{ulem}%
{略}%
{略}%
{
    以上命令分别输出(1)下划线、(2)双下划线、(3)虚下划线、(4)点下划线、(5)波浪线、(6)删除线、(7)斜删除线。
}%

\subsection{长度单位}
\notes
{
    \mnt{pt、pc、in、bp、cm、mm、sp、em、ex、\textwidth、\linewidth}
}%
{基础命令}%
{略}%
{略}%
{
    \LaTeX{}的长度单位:

    \textbf{pt}:point,磅

    \textbf{pc}:pica,1 pc = 12 pt,四号字

    \textbf{in}:inch,英寸,1 in = 72.27 pt

    \textbf{bp}:bigpoint,1 bp = 1/72 in

    \textbf{cm}:centimeter,厘米,1 cm = 1/2.54 in

    \textbf{mm}:millimeter,毫米,1 mm = 1/10 cm

    \textbf{sp}:scaled point,\TeX{}的基本长度单位,1 sp = 1/65536 pt

    \textbf{em}:当前字号下,大写字母M的宽度

    \textbf{ex}:当前字号下,小写字母x的高度

    两个常用的长度宏:

    \textbf{\mnt{\textwidth}}:页面上文字的总宽度,即页宽减去两侧边距

    \textbf{\mnt{\linewidth}}:当前行允许的行宽
}%

\subsection{空格}
\notes
{
    \mnt{~}
}%
{基础命令}%
{略}%
{略}%
{
    \mnt{~}输出效果等同于一个空格,并且在此空格之后不会换行,这样可以使空格前后内容始终在同一行上。
}%

\subsection{换行和分段}
\notes
{
    \textcolor{red}{\textbackslash{}par}、\mnt{\\、\newline、\mbox{}}
}%
{基础命令}%
{略}%
{略}%
{
    分段有两种方式:(1)在两段之间空一行、(2)在两段之间使用\textcolor{red}{\textbackslash{}par}命令,新的段落开头空两格打印。

    强制换行有两种方式:(1)在换行处使用\mnt{\\}命令、(2)在换行处使用\mnt{\newline}命令,下一行顶格打印。

    打印一个空白段落的方式:在空白段落处输入\mnt{\mbox{}}命令。注意空白段落和前后段落之间也要有空行或者\textcolor{red}{\textbackslash{}par}命令。该命令还有一个功能是输入参数,防止放入其中的词在换行时断开。
}%










\end{document}



% 代码知识
% 命令:
% \usepackage{minted}
% \newcommand{\mnt}[1]{\mintinline{latex}{#1}}
%* 注释:
%* 这一命令如果遇到\par这种LaTeX原生命令(primitive)作为参数,就会报错,这是因为通过\newcommand{}[]{}定义的宏命令当中的参数只接受普通宏命令,原生命令\par在被作为参数输入之前就会被LaTeX执行,从而导致报错;如果直接使用\mintinline{latex}{\par},结果反而不会报错。针对这一局部性的问题有一个解决办法,就是使用xparse宏包提供的\NewDocumentCommand{}{}{}来定义。
% 命令:
% \usepackage{xparse}
% \NewDocumentCommand{\mnt}{v}{\mintinline{latex}{#1}}
%* 注释:
%* xparse宏包提供比LaTeX原生\newcommand更强大、更灵活的宏定义机制。v = verbatim,不仅接受普通宏命令作为参数,而且也接受\par这种LaTeX原生命令作为参数。但是这一命令也存在限制:其无法嵌套在其他带参数的宏命令当中,本文档定义了\notes{}{}{}{}{}的宏命令,在其中使用通过上述方式定义的\mnt{}命令会报错。
%rfr xparse宏包官方文档的原文提到:A command with a verbatim argument will produce an error when it appears within an argument of another function.
%* xparse宏包解决的问题是,现在定义\mnt{}命令可以接受\par这种原生命令作为参数了,但是不管是哪种定义方式,最后都要在另一个带参数的宏命令当中使用,此时不管是前一种方式的\mintinline{latex}{\latex},还是有一种方式的\mnt{\par},在\notes{}{}{}{}{}中使用都会报错。如果直接用\NewDocumentCommand{}{}{}来定义\notes{}{}{}{}{},将\mnt{}包含于其中,则其中的命令都将只能被直接打印出来,无法被执行,但我们仍然希望可以自由选择某些命令被执行与否,以便观察其效果。结论就是:xparse在此处虽然部分地解决了问题,但是最终无法完全解决问题,一开始通过\newcommand{}[]{}定义的\mnt{}仍然是最直接的做法。至于涉及在\notes{}{}{}{}{}当中打印“\par”的源代码时,只能选择用直接表示的办法。

% 以下列举文档里碰到的LaTeX原生命令,随时补充,文档中用红色标出
% \par

