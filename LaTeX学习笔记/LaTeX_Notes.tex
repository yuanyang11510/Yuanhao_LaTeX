\documentclass{article}
% \usepackage[scheme = chinese]{ctex}
\usepackage[scheme = plain]{ctex}

\usepackage{indentfirst}
\usepackage{mdframed}
\usepackage{minted} % 打印源代码
\setminted{escapeinside = ||} % 设置转义定界符,||之间的内容转到LaTeX编译
\newcommand{\mnt}[1]{\mintinline{latex}{#1}} % 简化打印源代码命令
%* 上述命令存在限制,见文末注释
% 以下列举文档里碰到的LaTeX原生命令,随时补充,文档中用红色标出
\newcommand{\mntpar}{\textcolor{red}{\textbackslash{}par}}
\usepackage{ulem}
\usepackage{lettrine}
\usepackage[colorlinks = true]{hyperref}

% 命令介绍架构
\newcommand{\notes}[6]{
    \begin{mdframed}
        {\begin{center}\large\textcolor{red}{#1}\end{center}}
        \paragraph{命令}~\vspace{10pt}\\#2
        \paragraph{宏包}#3
        \paragraph{位置}#4
        \paragraph{参数}#5
        \paragraph{注释}#6
    \end{mdframed}
}

\title{\LaTeX{}学习笔记}
\author{殷元昊}
\date{}

\begin{document}
\maketitle

\section{前言}
这份笔记的前身来自笔者学习吴康隆的《简单高效LATEX》的笔记。笔者按照本书的结构学习了大部分代码,在学习的同时进行了大量的随文注释。但是在回看的时候,这些代码和随文注释混杂在一起,极大地影响了可读性。因此笔者深感有必要将这些随文注释整理出来,形成单独的\LaTeX{}文档。

在吴书之外,笔者还另外单独学习了一些有用的\LaTeX{}宏包,并也做了大量的随文注释,这些注释也需要整理出来。

值得一提的是,由于AI工具的兴起,代码学习的效率被大大提高,笔者在编辑本文档时,也仍然在使用AI学习新的知识,并立刻将其应用到本文档中

笔者曾经在学习制作表格和插入图片的相关宏包时制作过两份\LaTeX{}文档,但是回看的时候仍然觉得可读性不强,这是因为在编写文档的时候,尽管笔者使用了详细的文字描写相关命令,但由于缺少命令的参数结构,因而在回看时仍不免陷入“迷失在文字的海洋里”的尴尬境地,因此,该笔记将严格按照“宏包-位置-参数-注释”的架构搭建。

\section{正文}

\notes
{封面}%
{
    \mnt{\title{}、\author{}、\date{}、\thanks{}}
}%
{基础命令}%
{略}%
{略}%
{
    \begin{enumerate}
        \item \mnt{\date{}}如果省略,仍然会自动打印出编译当天的日期,如果不想显示日期,可以保留\mnt{\date{}},但不输入参数。
        %todo 但是还需要补充一个知识:如何通过该命令控制输出日期的格式
        \item 标题页的脚注用\mnt{\thamks{}}完成。
    \end{enumerate}
}%

\notes
{标题}%
{
    \mnt{\maketitle}
}%
{基础命令}%
{略}%
{略}%
{略}%

\notes
{目录}%
{
    \mnt{\tableofcontents、\listoffigures、\listoftables}
}%
{基础命令}%
{略}%
{略}%
{略}%

\notes
{目录标题}%
{
    \mnt{\contentsname、\listfigurename、\listtablename}
}%
{基础命令}%
{略}%
{略}%
{
    这三条命令用于\mnt{\renewcommand{}{}}结构中,通过重定义来修改目录的标题、图片目录的标题、表格目录的标题。
}%

\notes
{保留字符}%
{
    \mnt{(1) \# |\%| 输出|\#|}\\
    \mnt{(2) \$ |\%| 输出|\$|}\\
    \mnt{(3) \% |\%| 输出|\%|}\\
    \mnt{(4) \& |\%| 输出|\&|}\\
    \mnt{(5) \_ |\%| 输出|\_|}\\
    \mnt{(6) \{ |\%| 输出|\{|}\\
    \mnt{(7) \} |\%| 输出|\}|}\\
    \mnt{(8) \^{} |\%| 输出|\^{}|}
}%
{基础命令}%
{略}%
{略}%
{
    \begin{enumerate}
        \item 关于\%符号,吴书中提到:“如果在行末添加\%这个命令,可以防止LaTeX在行末插入一些奇怪的空白符”,其实这个说法语焉不详,我们知道,如果在LaTeX的代码中换一行,打印时就会在换行处插入一个空格,而此时如果在每一行的末尾插入一个\%符号,就能移除这个空格。
        \item 关于\^{}符号,该命令后面如果不加一对大括号,单独打印时会报错,而如果在大括号中填入一个字母(实际可以只填入一个字母而不需要大括号),输出的就是一个扬抑符(circumflex),比如\mnt{\^{a}}输出\^{a}。
    \end{enumerate}
}%

\notes
{保留字符:反斜杠}%
{
    (1) \mnt{\textbackslash}\\
    (2) \mnt{\textrm{\char92}、{\rmfamily\char92} |\% 使用ASCII码进行输出|}\\
    (3) \mnt{$backslash$ |\% 使用数学环境|}
}%
{基础命令}%
{略}%
{略}%
{
    \begin{enumerate}
        \item 方法二中的\mnt{\textrm{}}也可以替换为其他字体命令。
        \item 方法二、方法三输出的反斜杠和方法一输出的反斜杠不完全相同:如果将这几种方法输出的反斜杠排列在一行上,会发现方法一输出的反斜杠和后面反斜杠符号的间距更小,但是有意思的是,如果将反斜杠后面的符号替换成其他符号,这些反斜杠后的这一间距又会恢复相同,因此在实际情况中,只需要在需要连续使用两个反斜杠符号时,注意选择命令即可,比如,如果要输出“\textbackslash\textbackslash”符号,应该选用两个连续的\mnt{\textbackslash}命令。
        \item 顺便在此可以讨论\mnt{\texttt{}}和\mnt{{\ttfamily}}这两类不同的命令,这两类命令的大括号的位置和功能是不同的:前一种命令的大括号置于命令之后,是强制性的,用来放置相应的参数,只有大括号内的内容才会受到该命令的影响,其特点有点类似于环境。有些教程会将这种命令称为“argument style”(参数型)。
        
        而后一种命令的大括号不是强制性的,如果没有这一对大括号,则该命令之后的所有内容都会受到该命令的影响,如果需要限制该命令影响的范围,则可以在该命令和需要统辖的范围两边加上大括号。有些教程会将这种命令称为“declarative style”(声明型)。

        下文中会反复出现这两类命令,比如自定义距离命令、两类对齐方式命令等,以下只通过大括号的位置来体现这两类命令的类型,如果没有必要,不再就这一点展开具体讨论。
    \end{enumerate}
}%

\notes
{保留字符:波浪线}%
{
    (1) \mnt{\textasciitilde} \\
    (2) \mnt{\~{}}\\
    (3) \mnt{$sim$ |\% 实际是一个数学符号“约等于”(similar)|} 
}%
{基础命令}%
{略}%
{略}%
{
    \begin{enumerate}
        \item 方法一在不同的编辑器中输出效果不同,在TeXstudio中输出的是一个腭化符,而在VS Code中输出的却是一个正常的波浪线。
        \item 方法二和上文中的\^{}符号一样,该命令如果后面跟上一个字母,实际输出的就是一个波浪号/腭化符(tilde),如果要单独输出这一符号,必须在后面添加一对大括号,在TeXstudio中输出的是一个单独的腭化符,但是在VS Code中输出的就是一个正常的波浪线,比如\mnt{\~{a}}输出\~{a}。
    \end{enumerate}
}%

\notes
{大于号和小于号}%
{
    (1) \mnt{\textgreater、\textless}\\
    (2) \mnt{>、<}\\
    (3) \mnt{$>$、$<$}
}%
{基础命令}%
{略}%
{略}%
{
    \begin{enumerate}
        \item 文本中的大于号和小于号需要使用方法一中的\mnt{\textgreater}和\\\mnt{\textless}命令。
        \item 方法二和方法三输出的都是数学符号中的大于号和小于号。TeXstudio中直接输入大于号和小于号不会正确打印相应的符号,但是VS Code中正常,效果等同于两边加上数学环境。
    \end{enumerate}
}%

\notes
{引号}%
{
    \mnt{“你好,‘世界’!”} v.s. \mnt{``\thinspace`Max' is here.'' }
}%
{基础命令}%
{略}%
{略}%
{
    \begin{enumerate}
        \item 中文下的单引号和双引号可以用中文输入法直接输入。英文的左单引号是重音符“\mnt{`}”,右单引号是常用的引号符“\mnt{'}”。
        \item 吴书中提到上述“英文下的引号嵌套需要借助\mnt{\thinspace}命令分隔”,但实际上这和语言无关,\mnt{\thinspace}的效果是在命令处略微扩大外部引号和内部引号之间的距离,只是一个细节问题。
    \end{enumerate}
}%

\notes
{连字符、破折号和省略号}%
{
    (1) \mnt{- |\% 连字符:daughter-in-law|}\\
    (2) \mnt{-- |\% 数字起止符:1--2|}\\
    (3) \mnt{--- |\% 英文破折号:Listen---I'm serious.|}\\
    (4) \mnt{—— |\% 中文破折号|}\\
    (5) \mnt{…… |\% 中文省略号|}\\
    (6) \mnt{\ldots |~\% 英文省略号|} 
    %todo 该命令效果也等同于\dots,但是\ldots在任何模式下都适用,关于dots的相关命令有待探索
}%
{基础命令}%
{略}%
{略}%
{
    \mnt{\ldots} = lower dots, v.s. \mnt{\cdots} = center dots。注意连打三个句点“\mnt{...}”输出的不是真正的英文省略号:\mnt{\ldots}输出\ldots,“\mnt{...}”输出...。
}%

\notes
{强调}%
{
    \mnt{\emph{强调} v.s. \emph{empasis}}
}%
{基础命令}%
{略}%
{略}%
{
    中文的效果等同于\mnt{\textsl{}},西文的效果等同于\mnt{\textit{}}。
}%

\notes
{下划线和删除线}%
{
    \mnt{\underline{}}
}%
{基础命令}%
{略}%
{略}%
{略}%

\notes
{下划线和删除线}%
{
    (1) \mnt{\uline{} |\% \uline{下划线}|}\\
    (2) \mnt{\uuline{} |\% \uuline{双下划线}|}\\
    (3) \mnt{\dashuline{} |\% \dashuline{虚下划线}|}\\
    (4) \mnt{\dotuline{} |\% \dotuline{点下划线}|}\\
    (5) \mnt{\uwave{} |\% \uwave{波浪线}|}\\
    (6) \mnt{\sout{} |\% \sout{删除线}|}、\\
    (7) \mnt{\xout{} |\% \xout{斜删除线}|}
}%
{ulem}%
{略}%
{略}%
{
    “ulem”来自“ul (underline)”和“em (emphasis)”的合写。“sout”来自“strike out”的缩写。“xout”来自“cross out”的缩写。
}%

\notes
{长度单位}%
{
    \mnt{(1) pt |\% point,磅|}\\
    \mnt{(2) pc |\% pica,1 pc = 12 pt,四号字|}\\
    \mnt{(3) in |\% inch,英寸,1 in = 72.27 pt|}\\
    \mnt{(4) bp |\% bigpoint,1 bp = 1/72 in|}\\
    \mnt{(5) cm |\% centimeter,厘米,1 cm = 1/2.54 in|}\\
    \mnt{(6) mm |\% millimeter,毫米,1 mm = 1/10 cm|}\\
    \mnt{(7) sp |\% scaled point,\TeX{}的基本长度单位,1 sp = 1/65536 pt|}\\
    \mnt{(8) em |\% 当前字号下,大写字母M的宽度|}\\
    \mnt{(9) ex |\% 当前字号下,小写字母x的高度|}\\
    \mnt{(10) \textwidth |~\% 页面上文字的总宽度,即页宽减去两侧边距|}\\
    \mnt{(11) \linewidth |~\% 当前行允许的行宽|}
}%
{基础命令}%
{略}%
{略}%
{略}%

\notes
{空格}%
{
    \mnt{~}
}%
{基础命令}%
{略}%
{略}%
{
    \mnt{~}输出效果等同于一个空格,并且在此空格之后不会换行,这样可以使空格前后内容始终在同一行上。
}%

\notes
{换行和分段}%
{
    \mntpar{}、\mnt{\\、\newline、\mbox{}}
}%
{基础命令}%
{略}%
{略}%
{
    \begin{enumerate}
        \item 分段有两种方式:(1)在两段之间空一行、(2)在两段之间使用\mntpar{}命令,新的段落开头空两格打印。
        \item 强制换行有两种方式:(1)在换行处使用\mnt{\\}命令、(2)在换行处使用\mnt{\newline}命令,下一行顶格打印。
        \item 打印一个空白段落的方式:在空白段落处输入\mnt{\mbox{}}命令。注意空白段落和前后段落之间也要有空行或者\mntpar{}命令。该命令还有一个功能是输入参数,防止放入其中的词在换行时断开。
    \end{enumerate}    
}%

\notes
{段落间距}%
{
    \mnt{\parskip |\% 默认值:\the\parskip|}\\
    \mnt{\setlength{\parskip}{30pt} |\% 自定义段落距离|}
}%
{基础命令}%
{略}%
{略}%
{
    \begin{enumerate}
        \item 默认值可以通过\mnt{\the\parskip}得到。“0.0 pt plus 1.0 pt”是弹性距离,意即普通情况下是0pt,在有需要的时候可以拉伸到1pt,相应的也可以设置收缩距离(minus)
        \item “段落间距”其实是一个差值,是【前一段的最后一行的箱子和后一段的第一行的箱子的基线间距】减去【同一段内的基线间距】之后的差值,具体可以参见下文关于“基线间距”(baselineskip)的讨论。
        \item \mnt{\setlength{\parskip}{30pt}}如果放到导言区,会导致全文所有的段落间距变为30pt,直到设置新的段落间距为止。\\\mnt{{\setlength{\parskip}{30pt}}}如果放在正文当中,控制的对象是统辖范围内的每一段和其前面一段之间的距离(包括统辖范围内的第一段和统辖范围之前的最后一段之间的距离),因此,不管统辖范围内的最后一段的末尾是否空行或者加上\mntpar{}命令,其和统辖范围之后的第一段之间的距离都会恢复为原来的长度,因为这个段落间距是由统辖范围之后的第一段来控制的,下文中段首缩进长度、行距的设置的也具有类似的特点。
    \end{enumerate}
}%

\notes
{首字下沉}%
{
    \mnt{\lettrine |~\% 输出效果:~~\lettrine{T}{this} is an example.|}
}%
{lettrine}%
{略}%
{略}%
{略}%

\notes
{分页}%
{
    \mnt{\newpage}\\
    \mnt{\newpage\mbox{}\newpage |~\% 可以达到空出一整页的效果|}
}%
{基础命令}%
{略}%
{略}%
{
    \mnt{\newpage}命令的功能是从当前行开始到当前页的最后不再打印内容,而直接从下一页开始打印后续内容。
}%

\notes
{段首缩进}%
{
    \mnt{\parindent |~\% 默认值:\the\parindent|}\\
    \mnt{\setlength{\parindent}{10em} |\% 自定义段首缩进距离|}\\
    \mnt{\noindent |~\% 强制取消缩进|}
}%
{基础命令}%
{略}%
{略}%
{
    \begin{enumerate}
        \item 默认值可以通过\mnt{\the\parindent}得到。
        \item \mnt{\noindent}只会影响到其后的一个段落,而不会导致其后所有的段落都取消缩进,也不需要大括号来限制其统辖范围。
    \end{enumerate}
}%

\notes
{对齐方式}%
{
    \mnt{(1.1) \begin{flushleft}...\end{flushleft} |\% 左对齐|}\\
    \mnt{(1.2) \begin{flushright}...\end{flushright} |\% 右对齐|}\\
    \mnt{(1.3) \begin{center}...\end{center} |\% 居中对齐|}\\
    \mnt{(2.1) {\raggedright} |\% 左对齐|}\\
    \mnt{(2.2) {\raggedleft} |\% 右对齐|}\\
    \mnt{(2.3) {\centering} |\% 居中对齐|}\\
}%
{基础命令}%
{略}%
{略}%
{
    \begin{enumerate}
        % \item 左对齐的同时也就取消了段首缩进,事实上,右对齐和居中对齐同样也取消了段首缩进,只不过不在页面左侧,看不出来。
        \item “ragged”是“凹凸不平”的意思,以\mnt{\raggedleft}为例,其意思是文本的左侧是凹凸不平的,意即“右对齐”。
        \item (1)中都是参数型命令,(2)中都是声明型命令。吴书中提到,\mnt{\centering}(包括\mnt{\raggedleft}、\mnt{\raggedright})命令“常常用在环境内部(或者一对花括号内部)”,但这些都是“老旧的命令”,还是建议优先使用参数型命令。
        \item \mnt{\centering、\raggedleft、\raggedright}统辖范围内的最后一段后面需要空行或者加上\mntpar{}才能产生效果,否则最后一段的对齐方式将无法生效。或许正是这一容易出错的特点,使得这三个命令成为了不被推荐的命令。
        
    \end{enumerate}
}%

\notes
{字体:字族(family)}%
{
    \mnt{\familydefault |~\% 默认字族:\familydefault,“lmr”意即“Latin Modern Roman”|}\\
    \mnt{\textrm{}、{\rmfamily} |\% 罗马字族:\textrm{Roman}|}\\
    \mnt{\textsf{}、{\sffamily} |\% 无衬线字族:\textsf{Sans Serif}|}\\
    \mnt{\texttt{}、{\ttfamily} |\% 等宽字族:\texttt{Typewriter}|}\\
}%
{基础命令}%
{略}%
{略}%
{
    \begin{enumerate}
        \item \mnt{\familydefault}控制的是当前文档中的默认字族,一般就是\\\mnt{\rmdefault},而\mnt{\rmdefault}的默认值就是“lmr”,同理,也有\\\mnt{\sfdefault}和\mnt{\ttdefault},其默认值分别为“lmss”和“lmtt”。而\\\mnt{\textrm{}/{\tmfamily}}则是字体命令,会调用\mnt{\rmdefault}进行打印,其余两类字体命令同理。
        \item \mnt{\mintinline{}{}}中的字族会变为tt字族(如上文所示),在正文中,默认使用rm字族。
    \end{enumerate}
}%

\notes
{字体:字系(series)}%
{
    \mnt{\seriesdefault |~\% 默认字系:\seriesdefault|,“m”代表“Medium weight and width”}\\
    \mnt{\textbf{}、{\bfseries} |\% 粗体字系:\textbf{BoldSeries}|}\\
    \mnt{\textmd{}、{\mdseries} |\% 中粗体字系:\textmd{MiddleSeries}|}
}%
{基础命令}%
{略}%
{略}%
{
    \mnt{\mintinline{}{}}中的字系默认使用md字系,和正文中相同,但由于字族变为了tt字族,因此不容易看出bf字系和md字系的区别,此处分别给出rm字族和sf字族下搭配bf字系和md字系的区别:\\
    rm字族:\textrm{MiddleSeries v.s. \textbf{BoldSeries}}\\
    sf字族:\textsf{MiddleSeries v.s. \textbf{BoldSeries}}
}%

\notes
{字体:字形(shape)}%
{
    \mnt{\shapedefault |~\% 默认字形:\shapedefault|,“n”代表“Normal”}\\
    \mnt{\textup{}、{\upshape} |\% 竖直字形:\textup{Upright}|}\\
    \mnt{\textsl{}、{\slshape} |\% 斜体字形:\textsl{Slant}|}\\
    \mnt{\textit{}、{\itshape} |\% 强调体字形:\textit{Italic}|}\\
    \mnt{\textsc{}、{\scshape} |\% 小号大写体字形:\textsc{smallcapital}|}
}%
{基础命令}%
{略}%
{略}%
{
    \mnt{\mintinline{}{}}中的字形默认使用up字形,和正文中相同,此处分别给出rm字族和sf字族下搭配各个字形的区别:\\
    rm字族:\textup{Upright}、\textsl{Slant}、\textit{Italic}、\textsc{smallcapital}\\
    sf字族:\textsf{\textup{Upright}、\textsl{Slant}、\textit{Italic}、\textsc{smallcapital}}
}%


























\end{document}



% 代码知识
% 命令:
% \usepackage{minted}
% \newcommand{\mnt}[1]{\mintinline{latex}{#1}}
%* 注释:
%* 这一命令如果遇到\par这种LaTeX原生命令(primitive)作为参数,就会报错,这是因为通过\newcommand{}[]{}定义的宏命令当中的参数只接受普通宏命令,原生命令\par在被作为参数输入之前就会被LaTeX执行,从而导致报错;如果直接使用\mintinline{latex}{\par},结果反而不会报错。针对这一局部性的问题有一个解决办法,就是使用xparse宏包提供的\NewDocumentCommand{}{}{}来定义。
% 命令:
% \usepackage{xparse}
% \NewDocumentCommand{\mnt}{v}{\mintinline{latex}{#1}}
%* 注释:
%* xparse宏包提供比LaTeX原生\newcommand更强大、更灵活的宏定义机制。v = verbatim,不仅接受普通宏命令作为参数,而且也接受\par这种LaTeX原生命令作为参数。但是这一命令也存在限制:其无法嵌套在其他带参数的宏命令当中,本文档定义了\notes{}{}{}{}{}的宏命令,在其中使用通过上述方式定义的\mnt{}命令会报错。
%rfr xparse宏包官方文档的原文提到:A command with a verbatim argument will produce an error when it appears within an argument of another function.
%* xparse宏包解决的问题是,现在定义\mnt{}命令可以接受\par这种原生命令作为参数了,但是不管是哪种定义方式,最后都要在另一个带参数的宏命令当中使用,此时不管是前一种方式的\mintinline{latex}{\latex},还是有一种方式的\mnt{\par},在\notes{}{}{}{}{}中使用都会报错。如果直接用\NewDocumentCommand{}{}{}来定义\notes{}{}{}{}{},将\mnt{}包含于其中,则其中的命令都将只能被直接打印出来,无法被执行,但我们仍然希望可以自由选择某些命令被执行与否,以便观察其效果。结论就是:xparse在此处虽然部分地解决了问题,但是最终无法完全解决问题,一开始通过\newcommand{}[]{}定义的\mnt{}仍然是最直接的做法。至于涉及在\notes{}{}{}{}{}当中打印“\par”的源代码时,只能选择用直接表示的办法。

%* 另外需要注意,minted宏包提供了minted环境和mintinline命令,其中前者是不能放在其他带普通参数的命令中的因此本文档中,如果在\notes{}{}{}{}{}当中使用minted环境,会报错(但其他普通环境,比如enumerate环境,不会有这个问题)


