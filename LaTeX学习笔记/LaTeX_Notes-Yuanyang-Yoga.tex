\documentclass{article}
% \usepackage[scheme = chinese]{ctex}
\usepackage[scheme = plain]{ctex}

\usepackage{indentfirst}

\newcommand{\notes}[4]{
    \paragraph{参数结构}#1
    \paragraph{注释}#2
    \paragraph{注意事项}#3
    \paragraph{参考链接}#4
}
\newcommand{\tbsl}{\textbackslash}


\title{\LaTeX{}学习笔记}
\author{殷元昊}
\date{}

\begin{document}
\maketitle

\section{前言}
这份笔记的前身来自笔者学习吴康隆的《简单高效LATEX》的笔记。笔者按照本书的结构学习了大部分代码,在学习的同时进行了大量的随文注释。但是在回看的时候,这些代码和随文注释混杂在一起,极大地影响了可读性。因此笔者深感有必要将这些随文注释整理出来,形成单独的\LaTeX{}文档。

在吴书之外,笔者还另外单独学习了一些有用的\LaTeX{}宏包,并也做了大量的随文注释,这些注释也需要整理出来。

笔者曾经在学习制作表格和插入图片的相关宏包时制作过两份\LaTeX{}文档,但是回看的时候仍然觉得可读性不强,这是因为在编写文档的时候,尽管笔者使用了详细的文字描写相关命令,但由于缺少命令的参数结构,因而在回看时仍不免陷入“迷失在文字的海洋里”的尴尬境地,因此,该笔记将严格按照“参数结构-注释-注意事项-参考链接”的架构搭建,如果某些部分简单,则仍然写出对应的部分名称,后跟“略”。

\section{基础命令}
\subsection{封面}
\subsubsection{\tbsl{}title\{\}、\tbsl{}author\{\}、\tbsl{}date\{\}、\tbsl{}thanks\{\}}
\notes{略}%
{
    \begin{enumerate}
        \item \tbsl{}date\{\}如果省略,会自动以编译当天的日期为准,如果不想显示日期,可以保留\tbsl{}date\{\},但不输入参数
        %todo 但是还需要补充一个知识:如何通过该命令控制输出日期的格式
        \item 标题页的脚注用\tbsl{}thanks\{\}完成
    \end{enumerate}
}%
{略}%
{略}%





\end{document}