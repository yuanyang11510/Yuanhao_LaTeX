\documentclass{article}
\usepackage{ctex}
\usepackage{graphicx}
\usepackage{multicol}
\usepackage{caption}
\usepackage{blindtext}
\usepackage{cuted} % cuted宏包提供strip环境,用于解决figure*环境中的图片在双栏模式下无法在首页被插入的问题
%! 注意,在strip环境中需要使用caption宏包提供的"\captionof{float type}{heading}"命令来设置标题
% \usepackage{dblfloatfix} % dblfloatfix宏包用于解决figure*环境中的图片在双栏模式下无法在底部被插入的问题,但是仍然无法解决图片无法在首页被插入的问题
%! 注意,dblfloatfix宏包和stfloats宏包不兼容,只能加载其中一个
%* 但是dblfloatfix宏包可以和nidanfloat宏包兼容
\usepackage{stfloats} % stfloats宏包可以同时解决上述两个问题
% \usepackage{nidanfloat} % nidanfloat宏包也可以同时解决上述两个问题
%! 注意,加载nidanfloat宏包会导致通过cuted宏包提供的strip环境中的图片消失,但编译不会报错

\setlength{\columnseprule}{.4pt}

\begin{document}
% (一)在双栏模式下使用figure环境插入图片
\twocolumn
    \textbf{第一部分:figure环境}

    这是一段测试文字……这是一段测试文字……这是一段测试文字……这是一段测试文字……这是一段测试文字……这是一段测试文字……这是一段测试文字……这是一段测试文字……这是一段测试文字……这是一段测试文字……这是一段测试文字……这是一段测试文字……这是一段测试文字……

    \newpage 
    
    \begin{figure}[htbp]
        \includegraphics[width = .1\textwidth]{doraemon1.jpg}
        \caption{这是用figure环境在双栏模式下插入的图片(一)}
    \end{figure}
    这是一段测试文字……这是一段测试文字……这是一段测试文字……

    \begin{figure}[htbp]
        \includegraphics[width = .1\textwidth]{doraemon1.jpg}
        \caption{这是用figure环境在双栏模式下插入的图片(二)}
    \end{figure}
    这是一段测试文字……这是一段测试文字……
    \begin{figure}[htbp]
        \includegraphics[width = .1\textwidth]{doraemon1.jpg}
        \caption{这是用figure环境在双栏模式下插入的图片(三)}
    \end{figure}
    这是一段测试文字……这是一段测试文字……这是一段测试文字……

    %* 可以发现,在双栏模式下使用figure环境插入图片和在单栏模式下没有什么区别,只不过插入的区域从单栏变成了双栏,插入的图片可以在双栏内部浮动

% (二)在双栏模式下使用figure*环境插入图片
%! 由于加载了stfloats宏包,因此打印出的实际上是stfloats宏包的效果,如果想要查看在没有加载宏包的情况下,在双栏模式下使用figure*环境插入图片的效果,可以取消加载stfloats宏包
    \twocolumn
    %* 继续使用一次"\twocolumn[text]"命令,效果是另起一页继续双栏模式
    \textbf{第二部分:figure*环境}

    \begin{figure*}[htbp]
        \includegraphics[width = .1\textwidth]{doraemon1.jpg}
        \caption{这是用figure*环境在双栏模式下插入的图片(一)}
    \end{figure*}
    这是一段测试文字……这是一段测试文字……这是一段测试文字……

    \begin{figure*}[htbp]
        \includegraphics[width = .1\textwidth]{doraemon1.jpg}
        \caption{这是用figure*环境在双栏模式下插入的图片(二)}
    \end{figure*}
    这是一段测试文字……这是一段测试文字……
    \begin{figure*}[htbp]
        \includegraphics[width = .1\textwidth]{doraemon1.jpg}
        \caption{这是用figure*环境在双栏模式下插入的图片(三)}
    \end{figure*}
    这是一段测试文字……这是一段测试文字……这是一段测试文字……

    \blindtext[7]

    %* 以上做了两步操作:(一)将上一部分的双栏模式中的命令当中的figure环境全部替换为figure*环境,其他命令不变;(二)在原来的命令的末尾加上多行"\blindtext"命令,将双栏中的文本内容填充到三页的篇幅。经过这两步操作后,就会发现很有意思的打印效果:首先,所有的三张图片都被打印到了第二页或者第三页上,尽管这些图片其实很早就被插入进来,尤其是第一张图片,甚至在双栏模式开始之后的第一行就被插入进来,但是其仍然在第二页才被打印出来;其次,这三张图片都被插入进了第二页或者第三页的顶部,因此,可以发现figure*环境在双栏模式下其实并不遵循htbp中的hb规则,而只遵循tp规则(在第一页上似乎只遵循p规则,否则无法解释图片为什么没有被插入第一页的顶部);最后,这些图片被插入的位置并不处于双栏模式下,而是处于单栏模式下,类似于"\twocolumn[text]"命令可选参数的设置效果。结合以上三点,可以发现figure*环境的作用就是在双栏模式下插入【不在双栏内部浮动,而是在每一页的双栏顶部预留出的单栏区域浮动】的图片。这一环境有其存在的合理性:和在双栏模式下的figure环境相比,其可以保证大图在双栏模式下依然拥有足够的插入空间,与双栏模式之外的figure环境相比,其可以保证图片仍然可以插入到双栏文本的内部,而不是只能出现在双栏文本的前面或者后面,table*环境和table环境的区别也可以参照理解。
    %! figure*环境和table*环境中的"*"号和章节命令后面的"*"号代表了完全不同的功能,前者的功能上文已经详述,后者则会导致计数器停止计数,因此,需要意识到,LaTeX中的"*"号并不是一个带有专一功能的记号,而是一个代表在原有命令/环境的基础上改变某些功能的记号
    
%rfr 关于上述figure*环境的一些不足之处,多个宏包提出了解决的办法
%rfr (三)首先是关于figure*环境中的图片在双栏模式下只能在下一页被插入的问题,可以使用cuted宏包提供的strip环境来解决,可以参考:https://latex.org/forum/viewtopic.php?f=45&t=10661
\twocolumn
    % 以下strip环境由cuted宏包提供
    \begin{strip}
        \centering
        \includegraphics[width = .1\textwidth]{doraemon1.jpg}
        \captionof{figure}{这是用strip环境在双栏模式下插入的图片} %! 在strip环境中需要使用caption宏包提供的"\captionof{float type}{heading}"命令来设置标题
    \end{strip}

    \textbf{第三部分:strip环境}

    \blindtext[2]

%rfr (四)关于figure*环境中的图片在双栏模式下只能在双栏顶部被插入的问题,可以使用dblfloatfix宏包来解决("dbl"来自"double (column)"的缩写),但是仍然无法解决图片无法在首页被插入的问题,可以参考:https://tex.stackexchange.com/questions/167186/figure-environment-skips-page-while-using-two-column-document
%! 由于dblfloatfix宏包和stfloats宏包不兼容,因此在此文档中没有加载dblfloatsfix宏包,打印出的实际上是stfloats宏包的效果,如果想要查看dblflatsfix宏包的效果,可以选择加载该宏包而不加载另外两个宏包
\twocolumn
    \begin{figure*}[b] % 强制设置插入位置为底部
        \includegraphics[width = .1\textwidth]{doraemon1.jpg}
        \caption{这是加载了dblfloatfix宏包后,用figure*环境在双栏模式下插入的图片}
    \end{figure*}

    \textbf{第四部分:dblfloatfix宏包}

    \blindtext[4]

% rfr (五)有两个宏包可以一次性解决上述两个问题:stfloats宏包("st"似乎是来自"standard"的缩写)和nidanfloat宏包(这是一个日本人开发的宏包,"nidan"来自日语“二段”的罗马字,“二段”即指“双栏”),可以分别参考:https://blog.csdn.net/zhuang19951231/article/details/79176298、https://tex.stackexchange.com/questions/167186/figure-environment-skips-page-while-using-two-column-document
%! 由于nidanfloat宏包会导致通过cuted宏包提供的strip环境中的图片消失,但编译不会报错,因此此文档中只加载stfloats宏包,而不加载nidanfloat宏包,如果想要查看nidanfloat宏包的效果,可以加载该宏包,取消加载stfloats宏包
\twocolumn
    \begin{figure*}[bp]
        % 当设置为h时,图片会插入到第二页的顶部,说明h是无效设置,事实上,从figure*环境的功能出发也很容易理解这一点,h设置完全可以由figure环境来承担
        % 当设置为t(b)(p)时,图片会插入首页的顶部,这相当于cuted宏包提供的strip环境的效果
        % 当设置为b(p)时,图片会插入首页的底部,这是dblfloatfix宏包无法达到的效果
        \includegraphics[width = .1\textwidth]{doraemon1.jpg}
        \caption{这是加载了stfloats/nidanfloat宏包后,用figure*环境在双栏模式下插入的图片(一)}
    \end{figure*}

    \textbf{第五部分:stfloats宏包}

    \blindtext[3]

    \begin{figure*}[tbp]
        \includegraphics[width = .1\textwidth]{doraemon1.jpg}
        \caption{这是加载了stfloats/nidanfloat宏包后,用figure*环境在双栏模式下插入的图片(二)}
    \end{figure*}
    
    \blindtext[3]

\onecolumn

%* (六)下面讨论在multicol宏包提供的multicols环境中插入图片的问题
%! 在multicols环境中,上述的很多命令都会失效:
%! (1)使用figure环境插入图片,虽然编译不会报错,但是图片实际不会被插入进多栏中,使用cuted宏包提供的strip环境插入图片,也会产生类似的效果,在multocols环境中,只能使用figure*环境插入图片
%! (2)加载nidanfloat宏包会导致multicols环境中使用figure*环境插入的图片消失,但编译不会报错,类似于其对strip环境中的图片产生的影响
% (3)加载stfloats宏包,能够顺利使用figure*环境在多栏中插入图片,但是图片无法插入到首页当中,而只能从第二页开始插入
% (4)加载dblfloatfix宏包,产生的也是和(3)中一样的效果,这倒是和dblfloatfix宏包本身提供的功能相符合
%* (5)事实上,就算不加载stfloats宏包和dbfloatfix宏包,(3)和(4)中的效果仍然存在,说明这是multicols环境自带的效果,具体情况请看下文注释

        \begin{figure}[t]
            \includegraphics[width = .1\textwidth]{doraemon1.jpg}
            \caption{这是在multicols环境外插入的图片(首页顶部)}
        \end{figure}

        \begin{figure}[b]
            \includegraphics[width = .1\textwidth]{doraemon1.jpg}
            \caption{这是在multicols环境外插入的图片(首页底部)}
        \end{figure}
        %rfr 如果要达到在多栏的首页插入图片的效果,只能在multicols环境外之前的位置插入图片(此时选择figure环境还是figure*环境都可以),如果设置t(b)(p),则插入首页的顶部,如果设置b(p),则插入首页的底部,可以参考:https://latex.org/forum/viewtopic.php?f=45&t=10661
\begin{multicols}{3}[\centering 这是分成三栏的前言] 
        % 和"\twocolumn[text]"命令的可选参数一样,相应内容会打印在多栏顶部预留出的单栏区域

        \begin{figure*}[t]
            %* 当设置为t(b)(p)时,图片会从第二页开始插入合适页面的顶部,这也是默认情况下的效果
            \includegraphics[width = .1\textwidth]{doraemon1.jpg}
            \caption{这是在multicols环境中用figure*环境插入的图片(首页以后的顶部)}
        \end{figure*}

        \textbf{第六部分:multicols环境}

        \blindtext[2]

        \begin{figure*}[b]
            %* 当设置为b(p)时,图片会从第二页开始插入合适页面的底部,这相当于dblfloatfix宏包提供的效果
            \includegraphics[width = .1\textwidth]{doraemon1.jpg}
            \caption{这是在multicols环境中用figure*环境插入的图片(首页以后的底部)}
        \end{figure*}

        \columnbreak

        \mbox{}

        \columnbreak

        \blindtext[2]

\end{multicols}


\blindtext

\end{document}