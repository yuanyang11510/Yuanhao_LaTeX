\documentclass{article}
\usepackage{ctex}

\begin{document}

\rule[-2em]{5em}{2em}%
p%
\rule{5em}{2em}%
p%
\rule[2em]{5em}{2em}%
    % 可选参数用来调整升降高度(类似"\raisebox{raise}[height][depth]{text}"命令中的必选参数),必选参数分别是宽度和上下边距
    %! 但是"\raisebox{raise}{text}"命令中的必选参数可以使用"\width"、"\height"、"\depth"、"\totalheight"等命令,而"\rule[lift]{width}{thickness}"命令的可选参数中却不行,经检验,可以使用"\baselineskip"命令
    %* 可以发现,如果不设置升降高度的可选参数,标尺箱子的下边线和所在行的箱子基线重合

{\setlength{\fboxrule}{.1pt}
\setlength{\fboxsep}{0pt}
\fbox{标尺箱子}

% 将高度设置为0pt(记得带单位),可以将同一行上的两个文本隔开指定宽度的距离
\fbox{标尺箱子}\fbox{\rule{3em}{0pt}}\fbox{标尺箱子}

% 将宽度设置为0pt(记得带单位),可以将相邻两行文本隔开一定高度的距离
\fbox{\rule{0pt}{15.1298pt}}\fbox{标尺箱子}

\fbox{\rule{0pt}{15.1299pt}}\fbox{标尺箱子}
}
    %* 在竖直方向上,情况其实可以很复杂,因为增加上下边距到一定的值,必然会影响到基线间距(可以参见"5.3.3 行距"一节中关于基线间距的详细讨论),由于此处标尺箱子在默认情况下的底边和所在行的箱子基线是重合的,就可以利用这一命令来大致估算ctexart文档类下相应字符所在箱子的高度和深度,以此处“标尺箱子”四个汉字为例,当标尺箱子的上下边距增加到15.1298pt(精确到0.0001)时,还不会引起基线间距的变化,当增加到15.1299pt时,就会因为所在行的箱子的上下间距超过规定的基线间距而导致LaTeX强制为这一行规定了一个新的基线间距:箱子的上下间距加上1pt,打印的效果表现为如果继续增加标尺箱子的上下边距,当前所在行的箱子将会不断往下移动,并且标尺箱子的上边线和上一行箱子的下边线之间总是存在1pt的距离(来自"\lineskip"命令),这样,我们就得到了几个距离的近似值:ctexart文档类下基线间距的默认值16.44145pt(通过"\the\baselineskip"命令得到)、汉字“好”所处箱子的上下边距9.593pt(参见"5.3.3 行距"一节,暂时认为这就是所有汉字所处箱子的上下边距)、从下一行的箱子基线到上一行箱子底线的距离15.1298pt,我们就可以估算出汉字所处箱子的深度为16.44145-15.1298=1.31165pt,高度为9.593-1.31165=8.28135pt,高度和深度的比大致是8.28135/1.31165=6.314 : 1,深度大约占上下边距的1.31165/9.593=13.7%,高度大约占上下边距的8.28135/9.593=86.3%,当标尺箱子的上下边距增加到15.1299pt时,所在行的箱子的上下边距就变成15.1299+1.31165=16.44155pt,超过了基线间距的默认值16.44145pt,而此前标尺箱子的上下边距为15.1298p时,所在行的箱子的上下边距为15.1298+1.31165=16.44145pt,刚好等于基线间距的默认值16.44145pt,说明这个估算的精度是相对可靠的。最后,再回到一开始的说法,所谓“可以将相邻两行文本隔开一定高度的距离”,之所以是“一定高度”而不是“指定高度”,是因为这个高度有两种可能:第一种就是让相邻两行的基线间距仍然保持默认值,这通常不是这一命令追求的目的;另一种就是让相邻两行的基线间距超过默认值,粗略地说,就是让基线间距增大,精确点说,这要求设置的标尺箱子的上下边距加上箱子深度的默认值后超过基线间距的默认值,新的基线间距还要在这个距离和的基础上加上1pt

    %* 标尺箱子常见的参数设置方式是将可选参数设置为一个负数,其绝对值就是标尺箱子的深度,第一个必选参数宽度设置为0pt,第二个必选参数上下边距同时设定了标尺箱子的高度(上下边距-深度)。我们在"5.3.3 行距"一节提到,每一行的内容都可以看成处于一个大箱子内,有一条固定的基线(由这一行上处于默认状态的箱子的基线确定)穿过这个箱子,箱子顶部到基线的距离称为“高度”,箱子底部到基线的距离称为“深度”,现在我们还可以发现,其实每一行的大箱子内部还可以划分出许多小箱子,并且根据划分的依据不同,箱子的数量也可多可少,不过最重要的分别是这个大箱子内离基线最高的点和离基线最深的点,它们分别决定了整行大箱子的顶部(上边线)和底部(下边线),进而决定了整行大箱子的高度和深度。经过以上讨论,我们可以发现,标尺箱子的这种用法实际上是在为所在行最后的大箱子设置一个高度的最小值以及深度的最小值(当然前提是设置的高度和深度应该超过原来整行大箱子的高度和深度),此后不管这一行上的内容如何变化,整行箱子的高度和深度都不会低于这个标尺箱子的高度和深度。标尺箱子的一种常见用法是用来拉高表格的高度

{\setlength{\fboxrule}{.1pt}
\setlength{\fboxsep}{0pt}
\begin{tabular}{|c|}
    \hline
    \fbox{a}\\
    \hline
    \fbox{1}\\
    \hline
    \fbox{好}\\
    %* 如果不插入标尺箱子,表格的“行距”是默认值1
    %rfr “行距”由"\arraystretch"命令控制,可以参考文件"Table.tex" 
    \hline
    \rule[-2ex]{2pt}{6ex}\fbox{a1好}\\
    %* 插入设置的标尺箱子之后,表格的行距维持在6ex
    \hline
    \rule[-12pt]{2pt}{24pt}\fbox{a1好}\\ 
    %* 可以发现,一般将深度设置的比高度小一些,这样可以使所在行的文字处于中间偏下的位置
    \hline
    \rule[-2ex]{1em}{6ex}\fbox{\Huge a1好}\\ 
    %* 可以发现,由于临时增加了字体大小,导致表格的行距超过了原来标尺箱子设置的上下边距
    \hline
\end{tabular}
}
    %* "\strut"命令("strut"本身就是“支柱”的意思)插入的是一种特殊的标尺箱子,其效果等同于"\rule[-0.3\baselineskip]{0pt}{\baselineskip}",因此,这个特殊的标尺箱子其实是一个上下边距等于基线间距的箱子,深度为0.3倍的基线间距,高度为0.7倍的基线间距,由于这个标尺箱子的上下边距始终等于基线间距(随着基线间距的变化而变化),根据前文的估算,在ctexart文档类中,基线间距大致是16.44145pt,由此可以推算出字体大小大致是16.44145/1.2=13.7012083pt,汉字“好”所处箱子的上下边距大致是9.593pt,数字“1”所处箱子的上下边距大致是7.0193pt,小写字母所处箱子的上下边距大致是“a”4.838pt,其中,汉字所处箱子的高度为8.28135pt,占基线间距的比例约为8.28135/16.44145=50.3687%,深度为1.31165pt,占基线间距的比例约为1.31165/16.44145=7.9777%,其他字符以及其他文档类的情况应该也类似,即,基线间距应该大于相应的各类字符所处的箱子上下边距,各类字符所处箱子的高度小于0.7倍的基线间距,深度小于0.3倍的基线间距,这样,"\strut"命令才能发挥效果,让所在行整体所属的大箱子的高度和宽度分别不低于0.7倍的基线间距和0.3倍的基线间距
    %rfr 不过这一篇帖子却提出有些字符所处箱子的上下边距可能会超过基线间距:https://tex.stackexchange.com/questions/410250/understanding-line-height-line-spacing-baselineskip-in-latex,还需要进一步验证

% 可以观察下面几组命令的打印效果的不同,来体会"\strut"命令的“支柱”作用
    {\setlength{\fboxrule}{.1pt}
    \setlength{\fboxsep}{0pt}

% 不设置任何的“支柱”
    \framebox[3em]{好1a}%
    \framebox[3em]{好}%
    \framebox[3em]{1}%
    \framebox[3em]{a}%

    \framebox[3em]{好1a}%
    \framebox[3em]{好}%
    \framebox[3em]{1}%
    \framebox[3em]{a}%
    %* 可以发现,由于缺少支柱,箱子的上下边距将根据其中的字符大小自适应,使得箱子的上线边线和文本边框的上下边线的距离为0
    %% 此时的基线间距(实际上是段落间距,但是效果是相同的,下同)为默认值

% 使用一般的"\rule[lift]{width}{thickness}"命令将支柱距离设置为0.3倍的基线间距,其中深度为0.05倍的基线间距,高度为0.25倍的基线间距
    \framebox[3em]{\rule[-0.05\baselineskip]{2pt}{0.3\baselineskip}好1a}%
    \framebox[3em]{\rule[-0.05\baselineskip]{2pt}{0.3\baselineskip}好}%
    \framebox[3em]{\rule[-0.05\baselineskip]{2pt}{0.3\baselineskip}1}%
    \framebox[3em]{\rule[-0.05\baselineskip]{2pt}{0.3\baselineskip}a}%

    \framebox[3em]{\rule[-0.05\baselineskip]{0pt}{0.3\baselineskip}好1a}%
    \framebox[3em]{\rule[-0.05\baselineskip]{0pt}{0.3\baselineskip}好}%
    \framebox[3em]{\rule[-0.05\baselineskip]{0pt}{0.3\baselineskip}1}%
    \framebox[3em]{\rule[-0.05\baselineskip]{0pt}{0.3\baselineskip}a}%
    %* 可以发现,该标尺箱子设置的深度已经超过了数字“1”和小写字母“a”所处箱子原来的深度,因此这两个箱子的下边线到文本边框下边线的距离不为0,这两个箱子的下边线看起来就像是被“顶”了起来,并且处于同一条直线上
    %% 此时的基线间距仍然为默认值,因为包含汉字“好”的上下间距最大的箱子还没有被受到“支柱”的影响

% 使用一般的"\rule[lift]{width}{thickness}"命令将支柱距离设置为0.5倍的基线间距,其中深度为0.1倍的基线间距,高度为0.4倍的基线间距
    \framebox[3em]{\rule[-0.1\baselineskip]{2pt}{0.5\baselineskip}好1a}%
    \framebox[3em]{\rule[-0.1\baselineskip]{2pt}{0.5\baselineskip}好}%
    \framebox[3em]{\rule[-0.1\baselineskip]{2pt}{0.5\baselineskip}1}%
    \framebox[3em]{\rule[-0.1\baselineskip]{2pt}{0.5\baselineskip}a}%

    \framebox[3em]{\rule[-0.1\baselineskip]{0pt}{0.5\baselineskip}好1a}%
    \framebox[3em]{\rule[-0.1\baselineskip]{0pt}{0.5\baselineskip}好}%
    \framebox[3em]{\rule[-0.1\baselineskip]{0pt}{0.5\baselineskip}1}%
    \framebox[3em]{\rule[-0.1\baselineskip]{0pt}{0.5\baselineskip}a}%
    %* 可以发现,该标尺箱子设置的深度已经超过了所有箱子原来的深度,因此所有箱子的下边线到文本边框下边线的距离都不为0,并且这些箱子的下边线都处于同一条直线上;而该标尺箱子设置的高度超过了小写字母“a”所处箱子原来的高度,因此这个箱子的上边线到文本边框的上边线的距离不为0,这个箱子的上下边线都像是被“顶”了起来
    %% 此时的基线间距仍然为默认值,虽然所有箱子的下边线都因为“支柱”的影响而被“顶”了起来,但是仍然不足以影响基线间距,相邻两行箱子之间仍然存在自动补足的距离("glue")

% 使用"\strut"命令将支柱距离设置为基线间距
    \framebox[3em]{\strut 好1a}%
    \framebox[3em]{\strut 好}%
    \framebox[3em]{\strut 1}%
    \framebox[3em]{\strut a}%

    \framebox[3em]{\strut 好1a}%
    \framebox[3em]{\strut 好}%
    \framebox[3em]{\strut 1}%
    \framebox[3em]{\strut a}%
    %* 可以发现,通过"\strut"命令设置的标尺箱子的高度和深度都超过了所有箱子原来的高度和深度,因此所有箱子的上边线都处于同一直线上,下边线都处于同一直线上
    %% 此时是基线间距发生变化的临界点,由于上一行箱子的深度为0.3倍的段落间距,下一行箱子的高度为0.7倍的段落间距,加起来刚好等于段落间距,因此相邻两行的箱子应该是无缝堆叠在一起的,此处只是因为外面套的加框箱子存在0.1pt的边框宽度,导致系统在两个箱子之间自动插入了1pt的距离("\lineskip")

% 使用一般的"\rule[lift]{width}{thickness}"命令将支柱距离设置为1.5倍的基线间距,其中深度为0.3倍的基线间距,高度为1.2倍的基线间距
    \framebox[3em]{\rule[-0.3\baselineskip]{2pt}{1.5\baselineskip}好1a}%
    \framebox[3em]{\rule[-0.3\baselineskip]{2pt}{1.5\baselineskip}好}%
    \framebox[3em]{\rule[-0.3\baselineskip]{2pt}{1.5\baselineskip}1}%
    \framebox[3em]{\rule[-0.3\baselineskip]{2pt}{1.5\baselineskip}a}%

    \framebox[3em]{\rule[-0.3\baselineskip]{0pt}{1.5\baselineskip}好1a}%
    \framebox[3em]{\rule[-0.3\baselineskip]{0pt}{1.5\baselineskip}好}%
    \framebox[3em]{\rule[-0.3\baselineskip]{0pt}{1.5\baselineskip}1}%
    \framebox[3em]{\rule[-0.3\baselineskip]{0pt}{1.5\baselineskip}a}%
    %* 此时和"\strut"命令的情况同理,所有箱子的上下边线都在“支柱”的影响下被“顶”了起来,上边线处于同一条直线上,下边线处于同一条直线上
    %% 此时的基线间距变成了箱子的上下边距加上1pt,也就是1.5*16.44145+1=25.662175pt
    }
    %rfr 以上只是展示标尺箱子的“支柱”效果,还可以参考:https://latexref.xyz/_005cstrut.html,该网站提供了enumerate环境中运用"\strut"命令来增大段落间距的例子

\end{document}