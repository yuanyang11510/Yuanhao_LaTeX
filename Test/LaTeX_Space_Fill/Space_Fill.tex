\documentclass{ctexart}
\usepackage{blindtext}
\usepackage{lipsum}
\usepackage[showframe]{geometry}
\usepackage{lua-visual-debug} % 使用LuaLaTeX编译,可以将所有盒子和基线可视化
%! 下文当中很多关于“空格”的效果,都是在该宏包的帮助下发现的

\begin{document}

% 自定义水平距离
    a\hspace{2ex}b

    \hspace{5em}test test test test test test test test test test test test test test test test test test
    %* 在文本开头,"\hspace{l}"命令可以正常使用
    %% 注意,在开头使用"\hspace{l}"命令时,不要忘记系统同时会使用"\indent"命令,因此已经预先有一个向左缩进2em的距离,需要将"\hspace{l}"命令设置的距离在此基础上通过计算才能得到实际距离

% 下面讨论在“断行”处"\hspace{l}"命令和"\hspace*{l}"命令的不同表现
    %* “断行”其实仅指“自然断行”,但是下文同时讨论通过"\\"命令设置的强制断行以及通过"\par"命令设置的段落末尾的断行的情况,三种情况各自有一些不同的特点
% 自然断行
    {\setlength{\fboxsep}{0pt}
    \fbox{
        \begin{minipage}{.4\textwidth}
            test test test test test test test test test test test test test test
        \end{minipage}
        }
    \fbox{
        \begin{minipage}{.4\textwidth}
            test test test test test test test test\hspace{5em} test test test test test test
        \end{minipage}
        }
    }
    %* 可以发现,在minipage环境中,在自然断行处,"\hspace{l}"命令不起作用

    test test test test test test test test test test test test test test test test test test test test test test 

    test test test test test test test test test test test test test test test test test test test test\hspace{5em} test test 
    %* 在普通页面中,在自然断行处,"\hspace{l}"命令同样不起作用

    test test test test test test test test test test test test test test test test test test test test\hspace*{5em} test test 
    % 如果替换成"\hspace*{l}"命令,打印时就会插入对应距离的空格,并且文本排版自动调整
    %! 书中所说的“在(自然)断行处也正常输出空格”容易让人误以为是在原来的打印效果的基础上,在断行处打印出空格(实际上,这正是后面"\\"和"\par"两种情况下的效果),但实际情况是,使用"\hspace*{l}"命令之后,文本排版发生了自动调整
    %% 这反映出这一命令的用处实际上是在打印完毕之后在文本的自然断行处进行局部的微调,使整体的打印效果更加美观

% 通过"\\"命令设置的强制换行
    %* 此时还要区分是在"\\"命令的前面还是后面
    test test test test test test test test test test test test test test test test test\\ test test test test test 

    test test test test test test test test test test test test test test test test test\hspace{20em}\\ test test test test test
    %* 可以发现,如果"\hspace{l}"命令在"\\"命令的前面,不会发生作用
    %% 这涉及"\\"命令的底层定义:在"\\"命令的开头会执行"\unskip"命令(下文的"\par"命令同样如此),其效果是将前面的【第一段】填充距离(确切地说是水平填充距离,下同)/空格消除(也就是说如果是文本+空格+填充距离+\\,则"\\"命令只会消除填充距离而不会消除空格;如果是文本+空格+\\,则"\\"命令会消除空格),此处"\hspace{l}"命令设置的填充距离被其后的"\\"命令消除了,因此不会发挥作用

    test test test test test test test test test test test test test test test test test\\\hspace{20em} test test test test test 
    %* 可以发现,如果"\hspace{l}"命令在"\\"命令的后面,同样不会发生作用
    %% 注意,在"\\"命令前后的两个"\hspace{l}"命令,其不发挥作用的原理是不同的,上面已经介绍"\\"命令对于前面的填充距离/空格的处理,而在"\\"命令的后面,不管有多少段填充距离/空格,都会被自动忽略,因此强制换行后的文本总是从新的一行的开头开始
    %todo 至于【"\\"命令后面忽略所有的填充距离/空格】的原理,待研究

    test test test test test test test test test test test test test test test test test\hspace*{20em}\\ test test test test test 
    %* 可以发现,如果"\hspace*{l}"命令在"\\"命令的前面,就会在相应位置插入相应长度的空格(这是通过lua-visual-debug宏包观察到的,此后不再重复指出)
    %% 这说明"\hspace*{l}"设置的填充距离不会被其后的"\\"命令消除,这就是这一命令能够发挥作用的底层原因
    %todo 至于为什么"\par"命令不会消除"\hspace*{l}"命令设置的填充距离、这和"\hspace*{l}"命令本身底层定义有什么关系,待研究
    %! 注意,"\hspace*{l}"命令在强制断行处和在自然断行处的效果不同

    test test test test test test test test test test test test test test test test test\\\hspace*{20em} test test test test test 
    %* 如果"\hspace*{l}"命令在"\\"命令的后面,会在强制换行后的下一行开头插入相应长度的距离
    %% 可以发现,通过"\hspace*{l}"命令在"\\"命令的后面设置的填充距离不会被自动忽略忽略
    
% 通过"\par"命令设置的段落末尾断行
    %* 同样,还要区分是在"par"命令的前面还是后面
    test test test test test test test test test test test test test test test test test\par test test test test test 

    test test test test test test test test test test test test test test test test test\hspace{10em}\par test test test test test 
    %* 可以发现,如果"\hspace{l}"命令在"\par"命令的前面,不会发生作用
    %% 和"\\"命令一样,在"\par"命令的开头会执行"\unskip"命令,因此"\hspace{l}"设置的命令会被其后的"\par"命令消除

    test test test test test test test test test test test test test test test test test\par\hspace{10em} test test test test test 
    %* 在"\par"命令之后使用"\hspace{l}"命令等同于在下一段的文本开头使用该命令,因此会导致下一段文本的缩进距离变长
    %% 其实此时就又回到在文本开头使用"\hspace{l}"命令的情况

    test test test test test test test test test test test test test test test test test\hspace*{10em}\par test test test test test 
    %* 可以发现,如果"\hspace*{l}"命令在"\par"命令的前面,就会在相应位置插入相应长度的空格
    %% 同样的,这说明"\hspace*{l}"设置的填充距离不会被其后的"\par"命令消除

    test test test test test test test test test test test test test test test test test\par\hspace*{10em} test test test test test 
    %* 当"\hspace*{l}"命令在"\par"命令之后时,其效果和"\hspace{l}"命令没有区别

    %* 在除去(1)自然断行处、(2)"\\"命令的前后、(3)"\par"命令的前面这三处之外的地方使用"\hspace{l}"命令和"\hspace*{l}"命令的效果没有区别,都是在相应位置插入一段指定距离的空格
    %! 实际上,经过检验,上述这段文本的打印效果是在倒数第三个词处自然断行,此时如果在倒数第四个词后面插入一段长距离的空格,则再看插入距离的长短,如果比较短,则会在相应位置插入对应长度的空格,如果比较长,则只会让排版效果发生轻微的变化,并不会插入实际长度的空格(至于长和短的界限在哪里,待研究),而如果在倒数第五个词后面使用相同的命令,则会使得这段空格被插入,这段空格距离过长时,倒数第四个词可能超出文本区域(甚至整个纸张区域)。也就是说,如果在断行处的前一个空格处使用"\hspace{l}"命令,其效果和前面提到的两种效果都不相同,根据插入距离的长短会有不同的效果

\newpage

% 自定义竖直距离
    % 可以在两段之间使用
    {\setlength{\fboxsep}{0pt}
    \fbox{
        \parbox{2em}{
            \fbox{a}
            \par\vspace{2em}\par %* 可以省略其中的任何一个"\par"命令,下文会继续讨论
            \fbox{b}
        }
        }
    \fbox{\parbox{2em}{\rule{2pt}{2em}}}
    }
    
    % 也可以在同一段内的两行之间使用
    test\vspace{2em} test test test test test test test test test test test test test test test test test test test test test test 
    %* "\vspace{l}"命令放置在段落当中,效果是增加当前行和下一行之间的竖直距离

    % 注意比对下面使用"\vspace{l}"命令的两个例子
    test test test test test test test test test test test test test test test test test test test test test test test\vspace{2em} 

    test test test test test test test test test test test test test test test test test test test test test test test

    \vspace{2em}test test test test test test test test test test test test test test test test test test test test test test test
    %* 可以发现,这两个例子分别代表了"\vspace{l}"命令位于"\par"命令之后和之前的两种情况:第一个例子是"\vspace{l}\par",其效果是增加当前段落最后一行和后一段第一行之间的竖直距离;第二个例子是"\par\vspace{l}",效果和第一个例子相同(当然,我们也可以表述成“增加当前段落第一行和上一段最后一行之间的竖直距离”),这说明在两个段落之间(包括下一段的开头,这一点尤其要注意)使用"\vspace{l}"命令,效果和这一命令位于"\par"命令的前后没有关系,增加的都是前一段最后一行和后一段第一行之间的竖直距离
    %% 因此,如果在一个页面的第一段的开头使用"\vspace{l}"命令,该命令不会生效,因为此时不存在上一段,也就无从增加上一段的最后一行和下一段的第一行之间的竖直距离,对应到代码,就是此前不存在"\par"命令
    %rfr 注意,这一段竖直距离是在基线间距(其实是段落间距)的基础上加上去的,并不意味着用这段竖直距离去替换基线间距,可以参考:https://tex.stackexchange.com/questions/463039/spacing-difference-when-using-boxes,根据这篇帖子的介绍,在基线间距的基础上加上这段距离并不会导致基线间距的突变
    %* 从这一点上来说,这一命令有点类似于"\rule[lift]{width}{thickness}"命令的“支柱”作用,只不过"\vspace{l}"命令只能粗略地增加基线间距,而"\rule[lift]{width}{thickness}"命令可以更加详细地调节当前行大箱子的高度和深度,"\vspace{l}"一定会导致基线间距超过默认值,"\rule[lift]{width}{thickness}"命令则不一定,即使该命令追求的目的通常也是将当前行大箱子的高度和深度增加到超过默认值,最后使得基线间距也超过默认值

% 同理,存在与其对应的"\vspace*{l}"命令
    {\setlength{\fboxsep}{0pt}
    \fbox{
        \begin{minipage}{.3\textwidth}
            a
        \end{minipage}
        }
    }
    {\setlength{\fboxsep}{0pt}
    \fbox{
        \begin{minipage}{.3\textwidth}
            \vspace{\baselineskip}\par
            a
            \vspace{\baselineskip}\par
        \end{minipage}
        }
    }
    {\setlength{\fboxsep}{0pt}
    \fbox{
        \begin{minipage}{.3\textwidth}
            \vspace*{\baselineskip}\par
            a
            \vspace*{\baselineskip}\par
        \end{minipage}
        }
    }
    %* 可以发现,在minipage环境中,这两个命令都可以在第一段的前面和最后一段的后面加上一段竖直距离
    %% 这说明在minipage环境的底层定义中,在其中文本开始前一定存在一个类似"\par"命令的部分,这样,才可以使段前的"\vspace{l}"命令生效
\newpage
    \vspace{\baselineskip}\par
    a
\newpage
    \vspace{3\baselineskip}\par
    a
\newpage
    \vspace*{3\baselineskip}\par
    a
    %* 可以发现,在普通页面中,位于第一段前面的"\vspace{l}"命令不起作用,但是"\vspace*{l}"可以在第一段的前面加上一段竖直距离
    %% 前文已经提到,"\vspace{l}"命令不生效是因为第一段前不存在"\par"命令,而"\vspace*{l}"命令却可以生效,说明其底层定义当中一定存在一个设置类似"\par"命令的部分,类似前文提到的minipage环境
\newpage
    \blindtext\par\blindtext\par\blindtext\lipsum[1][1-13] text text
    %* 这一步是为了构造一个刚好占满一页上所有行的文本
\newpage
    \blindtext\par\blindtext\par\blindtext\lipsum[1][1-13]%
    \vspace{10ex}\blindtext
    %* 可以发现,在普通页面中,位于当前页面最后一行上的"\vspace{l}"命令不起作用
\newpage
    \blindtext\par\blindtext\par\blindtext\lipsum[1][1-13]%
    \vspace*{10ex}\blindtext
    %* 而位于当前页面最后一行上的"\vspace*{l}"命令会发挥作用,并且会导致文本排版的自动调整
    %! 当前页面的最后一行的位置类似于前文介绍水平填充距离时的段内自然断行处

\newpage

% "\fill"命令经常用作"\hspace{l}"命令或者"\vspace{l}"命令的参数,用来在相应的文本之间填充一段自动计算后得到的距离,使得文本间距在水平或者竖直方向上均匀分布
%* 下面分别展示"\hspace{\fill}"命令、"\hspace*{\fill}"命令、"\hfill"命令以及"\hfil"命令在文本的不同位置对应的打印效果,并且尝试导致这些效果的原因
    %rfr 以下关于四条命令之间不同效果的解释,参考了以下几篇帖子:https://tex.stackexchange.com/questions/183577/implicit-hfil-at-the-end-of-each-paragraph、https://ask.latexstudio.net/ask/question/3690.html
    %todo 不过其实还没有完全弄懂最底层的逻辑,待研究
    %! 理解填充距离命令时,有两个特殊位置时是需要注意的,即一行的开头和一行的末尾,下面的解释当中会反复涉及这两个特殊位置。
%* (一)在文本开头
    \hspace{\fill}b\par
    \hspace*{\fill}b\par
    \hfill b\par
    \hfil b\par
    %* 前三条命令的效果是一样的,都是在文本的前面填充一段距离,使得"b"位于当前行的最后;"\hfil"命令也在文本的前面填充一段距离,但只有"\hfill"命令填充距离的一半,使得"b"位于当前行的中间
    %% 第四条命令中,"\hfil"命令之所以只填充了"\hfill"命令的一半,涉及"\par"命令的底层定义,"\par"命令中含有"\parfillskip"这一命令,其长度为0pt plus 1fil,可以简单地理解为相当于一个"\hfil"命令,而"fil"在和"fill"(注意,存在"\fill"命令,但不存在"\fil"命令)处于同一行时会失效(用Knuth在TeXbook第71页的话说,后者比前者更“强力”),因此在前三条命令中,文本后面的"\par"命令中的"fil"会因为文本前面的"fill"而失效,而在第四条命令当中,文本前后各存在一个"fil",这导致b位于当前行的中间
    %todo 关于"\parfillskip"命令的原理,还有待研究

\hrulefill

%* (二)在文本中间
    a\hspace{\fill}b\par
    a\hspace*{\fill}b\par
    a\hfill b\par
    a\hfil b\par
    %* 第二种情况只是在第一种情况的基础上,在相应的命令前面加上一段文本而已,本质上是相同的

\hrulefill

%* (三)同时在文本中间和最后
    a\hspace{\fill}b\hspace{\fill} % 一(1)

    a\hspace{\fill}b\hspace{\fill}\par % 一(2)
    a\hspace*{\fill}b\hspace*{\fill}\par % 二
    a\hfill b\hfill\par % 三
    a\hfil b\hfil\par % 四
    %* 第三种情况其实是在第二种情况的基础上,在后一段文本的后面再加上一条相应的命令
    %% 但是经过检验,我们发现,此时在"\hspace{\fill}"命令的后面空一行或者加上"\par"命令,导致的结果是不同的,而其他命令则没有这个分别,因此下面需要讨论五条命令,将有分别的这两条命令分别称为一(1)和一(2)
    %* 表面上看,第一(1)条命令、第二条命令和最后一条命令的效果是相同的,但是通过比较第二种情况和第三种情况下的这几条命令,可以发现,第一(1)条命令的效果和第二条命令都在原来文本的后面填充一段距离,使得"b"位于当前行的中间;而第一(2)条命令、第三条命令和最后一条命令一样,都没有造成打印效果的变化
    %! 第一(1)条命令的效果比较奇怪,目前还解释不了,下面只解释另外四条命令
    %% 此处涉及前文提到的"\par"命令底层定义当中的"\unskip"命令,以及"\hspace{l}"命令和"\hspace*{l}"命令在"\par"命令前的不同表现:在第一(2)条命令、第三条命令和最后一条命令当中,其中的"fill"或者"fil"都被后面的"\par"命令消除,因此其效果和第二种情况当中完全相同。而第二条命令中的"fill"并没有被后面的"\par"命令消除,继而这个"fill"导致"\par"命令当中的"fil"失效,因此文本的前后各存在一个"fill",导致b位于当前行的中间。
    %rfr 因此这篇帖子(https://ask.latexstudio.net/ask/question/3690.html)指出,如果在第一(2)条命令和第三条命令的基础上想要达到和第二条命令相同的效果,可以在"\hspace{\fill}"命令或者"\hfill"命令和"\par"命令之间插入一个"\mbox{}"命令设置的零宽度箱子,或者另外再插入一个同样的"\hspace{\fill}"命令或者"\hfill"命令即可
    %todo 至于为什么第一(1)条命令当中,"\hspace{\fill}"命令之后不使用"\par"命令而是选择空一行时,"fill"也没有被"\par"命令消除,待研究

\hrulefill

%* (四)在文本最后
    ab\hspace{\fill} % 一(1)

    ab\hspace{\fill}\par % 一(2)
    ab\hspace*{\fill}\par % 二
    ab\hfill\par % 三
    ab\hfil\par % 四
    %* 四条命令的效果看起来是一样的,但是经过第三种情况的讨论,我们知道,其实这些命令可以分为三类:一(1)为一类,二为一类,一(2)、三、四为一类,其效果等同于"ab\par"

\hrulefill

% "\vspace{\fill}"、"\vspace*{\fill}"、"\vfill"、"\vfil"四条命令也有一些类似的不同效果
%* (一)在文本开头
\newpage
    \vspace{\fill}\par b1
\newpage
    \vspace*{\fill}\par b1
\newpage
    \vfill\par b1
\newpage
    \vfil\par b1
    %* 第一条和最后两条的命令一样,都不起作用,第二条命令在文本前面填充一段距离,使得"b1"位于当前页面的底部
    %% 前文已经提到,由于页面当中第一段的前面不存在"\par"命令,因此"\vspace{\fill}"命令、"\vfill"命令以及"\vfil"命令都不起作用,只有"\hspace*{l}"命令能够生效

%* (二)在文本中间
\newpage
    a2\vspace{\fill}\par b2
\newpage
    a2\vspace*{\fill}\par b2
\newpage
    a2\vfill\par b2
\newpage
    a2\vfil\par b2
    %* 第二种情况其实是在第一种情况的基础上,在相应的命令前面加上一段文本。由于此时四条命令不再处于第一段的开头,因此都可以生效,前三条命令的效果相同,都是在"a2"和"b2"之间填充一段距离,使得"a2"和"b2"各自位于同一页面的顶部和底部;第四条命令虽然也在"a2"和"b2"之间填充一段距离,但是只有其他三条命令填充距离的一半,使得"b2"位于当前页面的中间
    %% 受上文"\parfillskip"命令的启发,我们也可以对此处的各条命令的效果做出解释:在"\newpage"命令的底层定义当中,实际上暗含了一个"fil",因此在前三条命令中,"\newpage"命令当中的"fil"因为同一页内的"\vspace{\fill}"命令、"\vspace*{\fill}"命令和"\vfill"命令当中的"fill"而失效。而在第四条命令当中,其实在"b2"的前后各存在一个"\vfil"命令,这使得"b2"位于当前页面的中间
    %rfr 实际上,上文提到的"\parfillskip"命令和此处的解释一开始都是受这一篇帖子的启发:https://www.zhihu.com/question/462017394

%* (三)同时在文本中间和最后("\newpage"命令之前)
\newpage
    a3\vspace{\fill}\par b3\vspace{\fill}
\newpage
    a3\vspace*{\fill}\par b3\vspace*{\fill}
\newpage
    a3\vfill\par b3\vfill\
\newpage
    a3\vfil\par b3\vfil
    %* 第三种情况其实是在第二种情况的基础上,在后一段文本的后面再加上一条相应的命令
    %* 前三条命令的效果相同,都是在"a3"和b3"之间填充一段距离,"b3"之后填充一段距离,使得"a3"位于当前页面的顶部,"b3"位于当前页面的中间;第四条命令也在"a3"和b3"之间填充一段距离,"b"之后填充一段距离,但是最后"a3"位于当前页面的顶部,"b3"却位于当前页面的中间偏上的位置,
    %% 经过第二种情况的讨论,我们知道,在前三条命令当中,"\newpage"命令当中的"fil"因为前面的"\vspace{\fill}"命令、"\vspace*{\fill}"命令和"\vfill"命令当中的"fill"而失效,因此b3的前后其实各存在一个相同的竖直填充距离命令,使得"b3"位于当前页面的中间。而在第四条命令当中,"a3"和"b3"之间存在一个"\vfil"命令,而"b3"后面除了明确写出的"\vfil"命令外,再加上"\newpage"命令当中的"fil",实际上存在两个"fil",因此页面高度按照1:2的比例进行分割,"b3"实际位于距离页面顶部1/3的位置
    %! 上文提到,"\vspace{l}"命令在当前页面的最后一行上无法生效,而此处"\vspace{\fill}"、"\vspace*{\fill}"、"\vfill"、"\vfil"等命令在最后一段的后面都生效,这可以类比"\hspace{l}"命令在段内自然断行处不生效的情况,同理,"\vspace{l}"命令在当前页面的最后一行上(也就是【自然分页处】)不会生效。但是"\hspace{l}"命令在"\\"命令和"\par"命令前同样不生效,原因是这两条命令的顶层定义当中,开头存在一个"\unskip"命令会清除前面的第一段水平填充距离/空格,而"\newpage"命令不存在这种类似的特性(待确认),因此上述四条竖直填充距离命令在"\newpage"命令前都没有被清除
    %todo 尚不清楚竖直方向是否存在类似水平方向上的"\unskip"命令,待研究

%* (四)在文本最后("\newpage"命令之前)
\newpage
    a4\par b4\vspace{\fill}
\newpage
    a4\par b4\vspace*{\fill}
\newpage
    a4\par b4\vfill
\newpage
    a4\par b4\vfil
\newpage
    %* 经过第二种情况的讨论,我们认为这四条命令在"b4"之后填充的竖直距离应该是相同的,都是当前页面的剩余高度

%todo 书中称"\hspace{\fill}"命令和"\hfill"命令(以及"\vspace{\fill}"命令和"\vfill"命令)的作用相同,经过以上讨论,这一论断大体没错,只是还需要弄明白"\hspace{\fill}"命令在明确写出来的"\par"命令前和在空一行前的效果为什么不同,只有在前一种情况下,"\hspace{\fill}"命令的效果才和"\hfill"命令相同:在"\par"命令前都不生效,而在后一种情况下,"\hspace{\fill}"在空一行前生效
\end{document}