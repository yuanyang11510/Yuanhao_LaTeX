\documentclass[twoside]{article}
\usepackage{blindtext}
\usepackage[switch,pagewise,modulo]{lineno}
    % 使用lineno宏包可以生成行号
    %! 有的时候编译一次可能会出现奇怪的效果,可以尝试再次编译
    % left表示行号在页面左侧(默认值),right表示行号在页面右侧,switch表示双边模式下行号在页面外侧,switch*表示双边模式下行号在页面内侧
    %* 经检验,即使在单边模式下设置switch(*)参数,行号出现的位置不变
    % running表示每页连续编号(默认值),pagewise表示每页行号从1计数 
    % modulo表示每五行显示
    % rfr 可以参考lineno宏包的官方手册:https://jp.mirrors.cicku.me/ctan/macros/latex/contrib/lineno/doc/ulineno.pdf

\begin{document}

%* 先举一个包括了两个宏包的基本命令"\linenumbers"和"\nolinenumbers"的例子
    \linenumbers[10] % 行号开始编号,可以设置可选参数,表示起始行号
    % 如果使用带"*"符号的版本,则表示从1开始编号,不可以带可选参数
    % 如果使用"\runninglinenumbers(*)"命令,则表示跨页连续编号
    %! 注意,如果宏包命令中设置了"pagewise"参数,"\linenumbers"命令的可选参数将失效,始终从1开始编号,但是"\runninglinenumbers"命令的可选参数仍然有效
    \blindtext

    \blindtext
    \nolinenumbers % 行号结束编号

\hrulefill

%* "\linenumbers"命令和"\nolinenumbers"命令的管辖范围完全相同,都可以分为之前和之后两种情况

%* (一)先看命令之后的情况,以"\linenumbers"命令举例,命令之后的文本至少需要是完整的一段(命令和文本之间是否空一行或者存在"\par"命令都不影响),命令的管辖范围以段为单位向后延伸,完整的一段的标准是在段落的最后空一行或者使用"\par"命令,上文当中,"\linenumbers"命令之后的第一段文本之后空了一行,因此属于管辖范围,会带上行号,但是第二段文本之后直接跟上了"\nolinenumbers"命令,因此不属于管辖范围,没有带上行号,如果在这一段文本和后面的"\nolinenumbers"命令之间空一行,或者插入一个"\par"命令,则也会带上行号
%% 因此,在使用"\nolinenumbers"命令时,放置的位置需要特别注意,一定要记得在即将结束编号的文本之后空一行使用该命令,或者在文本之后插入一个"\par"命令之后不空行使用该命令,否则该命令前的那一段文本将不会带上行号。上述例子当中"\nolinenumbers"命令的放置位置只是供举例说明使用,实际情况当中一般是不会这样放置的

%* 如果将这一环境当中的"\linenumbers"命令替换为"\nolinenumbers"命令,就是最常见的结束编号之后的状态,不再赘述

%rfr 官方手册举了一个类似下面的这个例子,此处的"\begingroup"命令和"\endgroup"命令的效果相当于一对大括号,用来限制命令的管辖范围,可以发现,由于此时"\blindtext"命令之后直接跟上"\endgroup"命令,导致这段文本在该范围内没有形成完整的一段,因此不属于"\linenumbers"命令的管辖范围,不会带上行号
    \begingroup
        \linenumbers
        \blindtext
    \endgroup

\hrulefill

%* (二)再看命令之前的情况,此时又可以分为两种情况,

%* (1)如果命令之后空一行或者存在一个"\par"命令,命令之前存在一段文本,并且该文本和该命令之间不存在空一行或者"\par"命令,则该命令也会作用于这一段文本,比如上文的第一个例子当中,"\nolinenumbers"命令之后空了一行,前面的两段文本之间空了一行,则从空行到该命令之间的第二段文本就属于该命令的管辖范围,不会带上行号

%rfr 如果将这种环境当中的"\nolinenumbers"命令换成"\linenumbers"命令,则紧跟在"\linenumbers"命令前的完整的一段文本会带上行号,由于这种情况实在少见,因此,官方手册称这种编号方式是“费解”(esoteric)的,实际上,整个第(二)种情况都是令人费解的,因为几乎不会在实际情况当中这样使用命令,但是在这里讨论这一种情况,是为了在出现问题时能够理解问题所在,毕竟总会存在遗漏空行的时候
    \blindtext
    \linenumbers

\dotfill

%* (2)如果命令之后是(一)当中的情况,命令之前有一段文本,且该文本和该命令之间没有空行或者不存在"\par"命令,则该命令也会作用于这一段文本,这种情况也十分罕见,比如下面“构造”出的一个例子
    \blindtext
    \linenumbers
    \blindtext

    \nolinenumbers
    %% 还可以发现,如果在"\(no)linenumbers"命令的前后不空行各加上一段文本,这两段文本会合为一段

\hrulefill

%* 鉴于上述使用"\(no)linenumbers"命令时可能存在的遗漏空行问题,官方手册建议使用(running)linenumbers(*)/(running)linenumbers[]环境,环境结束前在底层会自动添加一个"\par"命令,因此不需要考虑末尾空一行的问题
    \begin{linenumbers}
        \blindtext

        \blindtext
    \end{linenumbers}

\hrulefill

%* 但是,前文提到的第(二)种情况,在环境命令中也同样存在,请看下面的例子
    \blindtext
    \begin{linenumbers}
        \blindtext

        \blindtext
    \end{linenumbers}
    \blindtext
    %% 和"\linenumbers"命令的情况一样,如果在linenumbers环境的前面不空行输入一段文本,这段文本会和环境中的第一段文本合为一段,并且同样带上编号。不过环境后面不空行输入的一段文本则会另起一段,并且也不会带上行号

%todo 以上"\linenumbers"命令和linenumbers环境的种种表现都让人联想到文件"LaTeX_Wu_Kanglong.tex"当中提到的对齐环境(raggedleft/raggedright/centering)和"\fontsize{size}{skip}"命令的例子,在这些环境或者命令当中,也都需要在相应范围中的文本的末尾记得空一行或者添加一个"\par"命令来确保环境或者命令生效,尽管此处只有"\linenumbers"命令有这个问题,linenumbers环境不需要考虑这个问题,至于这是否反映出这些命令在底层存在共性,待研究

\hrulefill

% 使用"\resetlinenumber[N]"命令在某处把行号设置为某个数值,如果不带可选参数,则默认设置为1
%! 注意这一命令名称的末尾是不带"-s"的
    \begin{linenumbers}
        \resetlinenumber[10]
        \blindtext
    \end{linenumbers}

\dotfill

    \begin{runninglinenumbers}
        \resetlinenumber[10]
        \blindtext
    \end{runninglinenumbers}

\dotfill

    \resetlinenumber[10]
    \runninglinenumbers
    \blindtext
    %% 可以发现,这一命令在pagesize模式下不起作用,除非将其和"\runninglinenumbers"命令或者runninglinenumbers环境搭配使用,相当于设置行数命令的可选参数

\hrulefill

% 使用"\modulolinenumbers[N]"命令设置每N行显示行号,如果不带可选参数,则默认为每5行显示行号
    \modulolinenumbers[3]
    \linenumbers
    \blindtext

\hrulefill








\end{document}