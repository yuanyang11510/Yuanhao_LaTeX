\documentclass[a4paper,12pt]{article}%the arguments between square brackets are options/optional. Most standard LATEX classes will only accept 10pt, 11pt or 12pt font options. Some classes, such as scrbook(from "KOMA-Script") are capable of accepting any font size as an option.

\usepackage{xpatch}
\usepackage{xspace}
\usepackage[xindy]{imakeidx}
\usepackage{csquotes}
\usepackage{xcolor}
\usepackage{colortbl}
\usepackage{array}
\usepackage{rotating}
\usepackage{graphicx}
\usepackage{diagbox}
\usepackage{booktabs}

\usepackage{ctex}
%"\usepackage{ctex}" is designed for the display of Chinese characters, even in the list of tables under "LuaLaTex", the other two alternatives cannot achieve this latter aim either under "XeLaTex" or "LuaLaTex".
%The other two alternatives:
%\usepackage{xeCJK}
%\usepackage{CJKutf8}
%"\usepackage{CJKutf8}" is not compatible with "\usepackage{xeCJK}".

\usepackage{polyglossia}
\setmainlanguage{french}
\setotherlanguage{english}
\setotherlanguage[variant=uk]{english}
%When the "\setotherlanguage{english}" is turned into note, the compilation seems to risk making the program break down, so try to fix the wanted languages before the compilation.
\setotherlanguage{vietnamese}

\usepackage{fontspec}
%\setmainfont{Doulos SIL}
%Attention, there is a conflict between the "Doulos SIL" and many other codes, like the code "\emph", "\textbf{}", etc. The alternative is the font "Charis SIL".
\setmainfont{Charis SIL}

\usepackage{hyperref}
\usepackage[french]{varioref}
\usepackage[french]{cleveref}

\title{\textsf{The First File of LaTex}}
%The title does not figure by default
\author{YIN Yuanhao}
\date{}
\hypersetup{colorlinks=true, linkcolor=red}
%true: for impression, false: just for display

\makeindex[columns=1, intoc]

\newcommand{\inalco}{\textsc{inalco}\xspace}
\newcommand{\motclef}[1]{\textsc{#1}}
\newcommand{\oubien}[2]{#1/#2}
%\renewcommand{\emph}[1]{``\textit{#1}''}
\newenvironment{newitshape}{``\itshape}{''}

\begin{document}
\maketitle
%pay attention to the position of "\maketitle"(after "\begin{document}")

\begin{abstract}
	This is the abstract(only for the class ``article''.)
\end{abstract}

L'\inalco où j'ai passé ma vie de Master...

\section{Introduction}\label{sec:Introduction}
{\scshape Texte}\footnote{This is a footnote.}
%all the following text after "scshape" will be set in small capital(that is why this instruction is called "sc-shape"), but can be delimited by braces(attention the first brace is before "\scshape"). This command can be used to highlight the wanted words. c.f. \emph{...}

text v.s. \emph{text} under emphasis\index{emphasis}
%The code "\emph" is used for emphasising the wanted words by turned them into italics. c.f. {\scshape...}
%As mentioned above, the code "\emph" is not compatible with the font of "Doulos SIL". One possible solution is to turn the words wanted to be emphasized into another font firstly, e.g. with the code "\texttt{}" (the code "\textem{}" does not work somehow). But this solution is not very satisfactory. But this solution is not very satisfactory because it changes the basic font of the text. Another better solution is to define de font as "Charis SIL".

\emph{Pay attention to the font of the \emph{word} emphasized in its superior sentence emphasized.}

\begin{quote}
	This is a quotation\index{quotation|see{verse}}.
\end{quote}

\begin{quotation}
	This is a quotation.
	
	This is another quotation of another paragraph.
\end{quotation}

\begin{verse}
	text in verse\index{verse|see{quotation}}
\end{verse}

\section{Section2}

\subsection{Subsection}
\begin{itemize}
\item 1st item of ``itemize''
	\begin{itemize}
	\item another item of ``itemize''
		\begin{enumerate}
		\item 1st item of ``enumerate''
			\begin{description}
			\item[1st] item of ``description''
			\item[2nd] item
%Only the item of "description" can be partially set in boldface
			\end{description}
		\item 2nd item
		\end{enumerate}
	\item another item 	
	\end{itemize}
\item 2nd item
\end{itemize}
\subsubsection{Subsubsection}

\paragraph{Paragraph}

\subparagraph{Subparagraph1}

\subparagraph{Subparagraph2}

\clearpage

\section*{Section3 non-numbered}
{\LaTeX}
%the braces are necessary here so as to set a blank between the logo "LaTex" and the following text. 
(prononcer «latèk») est un langage et un système logiciel de composition de documents de qualité professionnelle aux nombreux avantages, particulièrement en comparaison des logiciels de traitement de texte.
%this is a comment: Ctrl+T/U for comment and decomment

\textbackslash \textasciicircum \textasciitilde \{\} \$ \# \&  \%

% Table generated by Excel2LaTeX from sheet '阳上字'
\begin{table}[htbp]
  \centering
  \caption{方言\index{方言}調查字表(節選)}
    \begin{tabular}{|lllcrcr|}\hline
    漢字&韻攝&開合&等&調類&韻&聲母\\\hline
    舅&流&開&三&上&有&羣\\
    臼&流&開&三&上&有&羣\\
    咎&流&開&三&上&有&羣\\\hline
    \end{tabular}%
  \label{tab:addlabel}%
\end{table}%
%the argument "{ccc}" is necessary for "\begin{tabular}"

\begin{table}[htbp]
\centering
\caption{表格二}
\begin{tabular}{p{0.1\textwidth}p{0.15\textwidth}p{0.2\textwidth}}\toprule
一厘米&兩厘米&三厘米\\\midrule
一厘米的長度&兩厘米的長度&三厘米的長度\\\bottomrule
\end{tabular}
\end{table}
%the "{ccc}" cannot be compatible with "{p{0.1\width}}..."
%"{p{1cm}}" is identified by LaTex as "illegal unit"

\begin{table}[htbp]
\caption{表格三}
This is a floating table
\begin{tabular}{*3{c}}
A&B&C\\
aaaaaa&&cccccc\\\cmidrule{2-3}
a&\diagbox{b}{bb}&c\\\cmidrule{1-2}
abc&bac&cab\\
acb&bca&cba\\
\end{tabular}
\end{table}
in the text.

\begin{table}[htbp]
\centering
\caption{表格四}
\renewcommand{\tabcolsep}{0pt}
\begin{tabular}{ccc}\toprule
A&B&C\\ \midrule
AA&BB&CC\\
AAA&BBB&CCC\\ \bottomrule
\end{tabular}
\end{table}

\begin{table}[htbp]
\centering
\caption{表格五}
\renewcommand{\tabcolsep}{10pt}
\begin{tabular}{ccc}\toprule
A&B&C\\ \midrule
AA&BB&CC\\
AAA&BBB&CCC\\ \bottomrule
\end{tabular}
\end{table}

\begin{table}[htbp]
\centering
\caption{表格六}
\begin{tabular}{c@{\hspace{1cm}}c@{}c}\toprule
A&B&C\\ \midrule
AA&b&c\\
AAA&b&c\\ \bottomrule
\end{tabular}
\end{table}

\begin{table}[htbp]
\centering
\caption{表格七}
\begin{tabular*}{.7\textwidth}{c@{\extracolsep{\fill}}c@{\extracolsep{\fill}}c}\toprule
A&B&C\\ \midrule
AA&BB&CC\\
AAA&BBB&CCC\\ \bottomrule
\end{tabular*}
\end{table}

\begin{table}[htbp]
\centering
\caption{表格八}
\resizebox{.8\textwidth}{!}{
\begin{tabular}{ccc}\toprule
A&B&C\\ \midrule
aaaaaaaaaaaaaaaa&bbbbbbbbbbbbbbb&cccccccccccccccccc\\
aaaaaaaaaaaaaaaaaaaaaaaaa&bbbbbbbbbbbbbbbbbbbbbbbbbbb&cccccccccccccccccccc\\ \bottomrule
\end{tabular}}
\end{table}

\begin{sidewaystable}[htbp]
\centering
\caption{表格九}
This is a floating table rotated 90 degrees
\begin{tabular}{ccc}\toprule
A&B&C\\ \midrule
aaaaaaaaaaaaaaaa&bbbbbbbbbbbbbbb&cccccccccccccccccc\\
aaaaaaaaaaaaaaaaaaaaaaaaa&bbbbbbbbbbbbbbbbbbbbbbbbbbb&cccccccccccccccccccc\\ \bottomrule
\end{tabular}
in the text.
\end{sidewaystable}
%The legend of the table will be rotated as well.

\begin{table}[htbp]
\centering
\caption{表格十}
\rotatebox{-90}{
\begin{tabular}{ccc}\toprule
A&B&C\\ \midrule
AA&BB&CC\\
AAA&BBB&CCC\\ \bottomrule
\end{tabular}}
\end{table}

\begin{table}[htbp]
\centering
\caption{表格十一}
\renewcommand{\arraystretch}{1.5}
\begin{tabular}{ccc}\toprule
A&B&C\\ \midrule
AA&BB&CC\\
AAA&BBB&CCC\\ \bottomrule
\end{tabular}
\end{table}

\begin{table}[htbp]
\centering
\caption{表格十二}
\renewcommand{\arraystretch}{0}
\begin{tabular}{ccc}\toprule
A&B&C\\ \midrule
AA&BB&CC\\
AAA&BBB&CCC\\ \bottomrule
\end{tabular}
\end{table}

\begin{table}[htbp]
\centering
\caption{表格十三}
\resizebox{!}{.04\textheight}{
\begin{tabular}{ccc}\toprule
A&B&C\\ \midrule
AA&BB&CC\\
AAA&BBB&CCC\\ \bottomrule
\end{tabular}}
\end{table}

\begin{table}[htbp]
\centering
\caption{表格十四}
\begin{tabular}{>{\scshape}cc>{\itshape}c}\toprule
A&B&C\\ \midrule
AA&BB&CC\\
AAA&BBB&CCC\\ \bottomrule
\end{tabular}
\end{table}

\begin{table}[htbp]
\centering
\caption{表格十五}
\newcolumntype{s}{>{\scshape}c}
\newcolumntype{i}{>{\itshape}c}
\begin{tabular}{ics}\toprule
A&B&C\\ \midrule
AA&BB&CC\\
AAA&BBB&CCC\\ \bottomrule
\end{tabular}
\end{table}

\begin{table}[htbp]
\centering
\caption{表格十六}
\begin{tabular}{ccc}\toprule
A&B&C\\ \midrule
AA&BB&CC\\
\cellcolor{lightgray}AAA&BBB&CCC\\ \bottomrule
\end{tabular}
\end{table}
%Apart from the package "colortbl", another package "xcolor" is needed for the definition of wanted colors.

\begin{table}[htbp]
\centering
\caption{表格十七}
\begin{tabular}{ccc}\toprule
A&B&C\\ \midrule
AA&BB&CC\\
\rowcolor{lightgray}AAA&BBB&CCC\\ \bottomrule
\end{tabular}
\end{table}

\begin{table}[htbp]
\centering
\caption{表格十八}
\begin{tabular}{c>{\columncolor{lightgray}}cc}\toprule
A&B&C\\ \midrule
AA&BB&CC\\
AAA&BBB&CCC\\ \bottomrule
\end{tabular}
\end{table}


\begin{table}[htbp]
\centering
\caption{表格十九}
\begin{tabular}{c>{\columncolor{lightgray}}cc}\toprule
A&\cellcolor{white}B&C\\ \midrule
AA&BB&CC\\
AAA&BBB&CCC\\ \bottomrule
\end{tabular}
\end{table}

/hə'ləu ðə wɜːld/ 

/vwasi lɛ̃trɔdyksjõ/ \ref{sec:Introduction} \nameref{sec:Introduction} (p. \pageref{sec:Introduction})

/adrɛs ɛ̃tɛrnɛt də linalko/ \url{http://www.inalco.fr}

/vwasi lɛ̃trɔdyksjõ/ \vref{sec:Introduction} \vpageref{sec:Introduction}
%The package "[french]{varioref}" does not seem to work, because the remark "ci-contre/page précédente" does not figure.

/vwasi la/ \cref{sec:Introduction}

Nous sommes le {\today} (fr)

Today is \textenglish[variant=uk]{\today} (uk) 

\begin{english}
Today is {\today} (am)
\end{english}

%"\foreignquote{english}{Today is \today}"does not work.

L'al\-lo\-cu\-ti\-vi\-té

%"\hyphenation{al-lo-cu-ti-vi-té}" does not display the text.

', --, ---, << >>, `` '', \dots, \textexclamdown, \textquestiondown
%Pay attention to the codes of the left and right quotation marks.

\textsuperscript{123abcABC} 123abcABC \textsubscript{123abcABC}

\'{o} \`{o} \^{o} \"{o} \v{o} \~{o} \u{o} 
\={o} \.{I} \H{o} \c{c} \d{o} \r{o} \b{o}
\k{o}

\'o \`o \^o \"o \~o \=o \.I

\OE \oe \AE \ae \ij \i \ss \SS \o \O \l \L
\aa \AA \dh \DH \dj \DJ \ng \NG

$\alpha\beta\gamma\delta\epsilon$

\textcopyright, \pounds, \$, \texteuro
\textrightarrow
$\emptyset, \neq, \simeq$

\motclef{abc}

\oubien{a}{b}

\begin{newitshape}
This is the new environment.
\end{newitshape}

\newcounter{compteur1}
\newcounter{compteur2}[compteur1]
%Attention the initial value is 0.

\stepcounter{compteur1}\arabic{compteur1}\Roman{compteur1}
	\stepcounter{compteur2}\arabic{compteur2}
	\stepcounter{compteur2}\arabic{compteur2}
%So the code "\stepcounter{}" ensures that the fist printed value will be 1.

\stepcounter{compteur1}\arabic{compteur1}\Roman{compteur1}
	\stepcounter{compteur2}\arabic{compteur2}
	\stepcounter{compteur2}\arabic{compteur2}

\stepcounter{compteur1}\arabic{compteur1}\Roman{compteur1}
	\stepcounter{compteur2}\arabic{compteur2}
	\stepcounter{compteur2}\arabic{compteur2}


\listoftables
\indexprologue{This is the list of indexes.}
\printindex

\end{document}
