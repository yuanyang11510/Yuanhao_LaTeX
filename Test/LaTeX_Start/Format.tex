\documentclass{article}

\usepackage{setspace}
\usepackage[top=3cm, bottom=3cm, left=2.5cm, right=2.5cm]{geometry}
\usepackage{fontspec}
\setmainfont[Mapping=tex-text, Numbers=OldStyle, Ligatures=Common]{Times New Roman}

\begin{document}
Texte \textrm{avec empattements}.
%font by default (for maintmatter)

Texte \textsf{sans empattements}.
%for title or perhaps for some structure elements

Texte \texttt{à chasse fixe}.\newline

Texte \textbf{en gras}.

\textbf{Texte \textmd{non-gras} dans du gras}.\newline

Texte \textit{en italique}.

\textit{Texte en \textup{style droit} dans de l'italique}.\newline

Texte \textsc{en petites capitales}.

Texte \uppercase{en capitales}

Texte \lowercase{EN MINUSCULES}\newline

{\tiny Texte
\scriptsize Texte
\footnotesize Texte
\small Texte
\normalsize Texte
\large Texte
\Large Texte
\LARGE Texte
\huge Texte
\Huge Texte}

\begin{center}
Texte centré.
\end{center}

\begin{flushleft}
Texte aligné à gauche.
\end{flushleft}

\begin{flushright}
Texte aligné à droite.\newline
\end{flushright}

Texte texte.

Texte \phantom{tex}te.\medskip%(\bigskip, \smallskip)

Texte \hspace{2em} texte.\newline

Unités de mesure (liste non exhaustive) :

— cm, mm, in ;

— pt : point, 0.3527 mm ;

— em : taille de la police de base (si 11pt, 1em = 11pt) ;

— ex : hauteur d’approximativement un x minuscule dans la police de \mbox{base ;}

— \textbackslash baselineskip : distance normale entre deux lignes d’un paragraphe ;

— \textbackslash textwidth : largeur du texte sur la page ;

— \textbackslash textheight : hauteur du texte sur la page ;

— \textbackslash parindent : largeur de l’alinéa.\\

Texte

{Texte
\doublespacing%onehalfspacing

Texte

Texte}\\

Texte\\texte\par Texte.
%The code "\par" has the same effect as two successive newlines.



\end{document}