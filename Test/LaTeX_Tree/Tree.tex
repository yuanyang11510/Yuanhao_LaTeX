\documentclass[11pt]{standalone}
\usepackage{ctex}
\usepackage[linguistics]{forest}
\usepackage{gb4e}

\begin{document}
%\forestset{default preamble={sn edges}}%但是在该系统中默认分叉处汇聚成一点
\newcounter{excompteur}
\newenvironment{exx}{%
\stepcounter{excompteur}%
(\arabic{excompteur})}{}

%1
\begin{exx}
	基础操作及隐去节点
	\begin{forest}
	[X[A][B]]
	\end{forest}
	
	\begin{forest}
	[[A][B]]%隐去节点内容但不隐去树枝,注意相关的中括号不能省去
	\end{forest}
\end{exx}

%2
\begin{exx}
	隐去树枝及该树枝上的所有节点
	\begin{forest}
	[X[A][B[][D]]]%必须有逗号,否则会输出文字"phantom"
	\end{forest}
	
	\begin{forest}
	[X[A][B[, phantom][D]]]%必须有逗号,否则会输出文字"phantom"
	\end{forest}
\end{exx}

%3
\begin{exx}
	使分叉处汇聚成一点
	\begin{forest}
	[[A][B][C][D]]
	\end{forest}
	
	\begin{forest}
	sn edges%该系统中此命令没有意义,因为默认分叉处汇聚成一点
	[[A][B][C][D]]
	\end{forest}
\end{exx}

%4
\begin{exx}
	调整端点的层级
	\begin{forest}
	[X[A, tier=1][B,tier=3[C,tier=1][D[E, tier=2][F]]]]%只看层级标记间的相对关系,层级标记可以是数字,也可以是字母
	\end{forest}
	
	\begin{forest}
	[X[A, tier=X][B,tier=A[C,tier=B][D[E, tier=X][F]]]]%只看层级标记间的相对关系,层级标记可以是数字,也可以是字母
	\end{forest}
	
	\begin{forest}
	where n children=0{tier=X}{}%待研究child和parent的意义
	[[A][B[C][D[E][F]]]]
	\end{forest}
\end{exx}

%5
\begin{exx}
	在隐去节点的情况下使被隐去节点的树枝平滑相连
	\begin{forest}
	[X[A][[C][D]]]
	\end{forest}
	
	\begin{forest}
	nice empty nodes%要求原B的位置没有内容(但不等于隐去)该命令才有意义
	[X[A][[C][D]]]
	\end{forest}
\end{exx}

%6
\begin{exx}
	调整树枝的角度
	\begin{forest}
	[[A][B[C][D]]]
	\end{forest}
	
	\begin{forest}
	[[A][B, calign=first[C][D]]]%c代表children
	\end{forest}
	
	\begin{forest}
	[[A][B, calign=last[C][D]]]
	\end{forest}
	
	\begin{forest}
	for tree={calign=first}
	[[A][B[C][D]]]
	\end{forest}
	
	\begin{forest}
	for tree={calign=last}
	[[A][B[C][D]]]
	\end{forest}
\end{exx}

%7
\begin{exx}
	调整树的角度
	\begin{forest}
	[X[A][B]]%默认向下生长
	\end{forest}
	
	\begin{forest}
	for tree={grow=north}%注意grow外面还有一个大括号
	[X[A][B]]%向上生长
	\end{forest}
	
	\begin{forest}
	for tree={grow=west}
	[X[A][B]]%向左生长
	\end{forest}
	
	\begin{forest}
	for tree={grow=east}
	[X[A][B]]%向右生长
	\end{forest}
	
	\begin{forest}
	for tree={grow'=south}%调换树枝的角度
	[X[A][B[C][D]]]
	\end{forest}
	
	\begin{forest}
	for tree={grow=north}%将树的角度和树枝的角度分别设置
	[X[A][B, grow'=west[C][D]]]
	\end{forest}
\end{exx}

%8
\begin{exx}
	调整树的形状	
	\begin{forest}
%	forked edges%该命令报错
	[[A][B[C][D]]]
	\end{forest}
\end{exx}

%9
\begin{exx}
	调整树枝的形状
	\begin{forest}
	[[A][B, edge=dashed[C, edge=dotted][D]]]
	\end{forest}
\end{exx}

%10
\begin{exx}
	用TikZ代码装饰树枝和节点
	\begin{forest}
	[[A][B, edge label={node[midway, above right]{b}}[C, draw][D, draw, circle]]]%注意node外面还有一个大括号;draw默认在节点外画方框
	\end{forest}
	
	%以下三个是更高级的命令
	\begin{forest}
	[ [A]
	[B, tikz={\node [draw, rounded corners,
	dashed, inner sep=0pt, fit to=tree] {};}
	[C] [D] ] ]
	\end{forest}
	
	\begin{forest}
	[ [A, name=A] [B [C] [D, name=D] ] ]
	\draw[->] (D.north) -- (A.south);
	\end{forest}
	
	\begin{forest}
	[ [A, name=A] [B [C] [D, name=D] ] ]
	\draw[->, dashed] (D) to [out=south west,
	in=south] (A);
	\end{forest}
\end{exx}

%11
%\begin{exx}
%	将forest和exe结合起来用
%	\begin{exe}%结果会报错,可能是standalone和exe不兼容所致
%	\ex
%	\begin{forest}
%	[[A][B[C][D]]]
%	\end{forest}
%	
%	\ex
%	\begin{forest}
%	[, baseline[A][B[C][D]]]
%	\end{forest}
%	\end{exe}
%\end{exx}

%待研究更复杂的命令,比如谱系树,句法树,音系树等等

\end{document}