\documentclass{article}
\usepackage{import} % import宏包提供了"\import{full path}{file}"命令和"\subimport{relative path}{file}"命令,前者类似"\input{}"命令,后者可以嵌套使用:在文件一中使用"\import{full path}{file}"命令引用文件二,在文件二中使用"\subimport{relative path}{file}"命令引用文件三,这样就可以直接在文件一中使用"\import{full path}{file}"命令引用文件三
%rfr 可以参考import宏包的官方手册:https://ftp.kddilabs.jp/CTAN/macros/latex/contrib/import/import.pdf,其中关于相对路径的说明:"The 〈full-path〉 argument for \import can be a full absolute path or a relative path starting from the main working directory for the document."

\usepackage{ctex}
\usepackage{geometry}
\usepackage{graphicx}
\usepackage{hyperref}

% \import{../LaTeX_Include/Package/}{Package.tex}

%! "\include{}"命令会使编译报错
% \usepackage{ctex}
\usepackage{geometry}
\usepackage{graphicx}
\usepackage{hyperref}

% \includeonly{Chapter_1.tex,Chapter_2.tex}
%rfr 这篇帖子(https://cfranc.wordpress.com/2009/12/01/include-input-and-relative-paths-in-latex/)当中有答案认为是MiKTeX的bug,但是经检验,即使安装了TeXLive并卸载了MiKTeX,问题仍然存在

\begin{document}

% "\import"命令
\import{../LaTeX_Include/Chapter1/}{Chapter_1.tex}
%* 嵌套命令可以使用"\import-\input",也可以使用"\import-\subimport",但是不能使用"\input-\input",也不能够使用"\import-\import"
% \import{../LaTeX_Include/Chapter2/}{Chapter_2.tex}
% \import{../LaTeX_Include/Forbidden/}{Forbidden.tex}

% "\input"命令
%! \section{一}
这是第一节

\subsection{(一)}
这是第一节下的次节(一)

\subimport{../Chapter1/Chapter1.2/}{Chapter1.2.tex}

%! 不要使用上面这条命令,因为引用的文件"Chapter_1.tex"中存在"\input{}"命令
\section{二}
这是第二节


\section{禁}
这是不被引用的一节



% "\include"命令
% "\include"命令会在引用文件前添加"\clearpage"命令进行分页
%! "\include{}"命令会使编译报错
% \section{一}
这是第一节

\subsection{(一)}
这是第一节下的次节(一)

\subimport{../Chapter1/Chapter1.2/}{Chapter1.2.tex}

% \section{二}
这是第二节


% \section{禁}
这是不被引用的一节



\end{document}