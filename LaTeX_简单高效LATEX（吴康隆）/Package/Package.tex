% 宏包-----------------------------------------------------------------------------------------------------------------------------------

% \usepackage{ctex}

\usepackage{lipsum}
            %rfr lipsum宏包提供"\lipsum"命令,用于生成一段无意义的测试文本,默认输出150段中的前7段,可以通过可选参数来指定输出的段落序数,比如:"\lipsum[4-57]"(第4段到第57段)、"\lipsum[23]"(第23段),还可以通过第二个可选参数指定输出段落中的句子序数,比如:"\lipsum[3-9][4-8]"(第3段到第9段当中,每一段的第4句到第8句),可以参看其官方手册:https://jp.mirrors.cicku.me/ctan/macros/latex/contrib/lipsum/lipsum.pdf
            %* 经检验,无法在一个指定参数中指定不连续的段落或者句子,比如"\lipsum[2,4]"、"\lipsum[2-4][1,3]"都无法输出正确的结果

\usepackage{blindtext}
            %rfr blindtext宏包提供"\blindtext"命令,生成一段无意义的测试文本,可以通过可选参数来指定重复次数(重复后的多段文本在一个段落内首尾相接);"\Blindtext"命令用于生成五段相同的上述文本,可以通过可选参数来控制重复的段落数,默认值为5,可以通过第二个可选参数来控制每个段落内文本的重复次数(相当于"\blindtext"命令的可选参数);"\blinddocument"命令用于生成一篇带有章节名(section、subsection……)、列表(itemize、enumerate、description)等内容的文档;"\Blinddocument"命令用于生成一篇更长的类似文档,可以参考其官方手册:https://jp.mirrors.cicku.me/ctan/macros/latex/contrib/blindtext/blindtext.pdf

% \usepackage{lua-visual-debug}
            % 使用LuaLaTeX编译,可以将所有盒子和基线可视化

\usepackage[normalem]{ulem}
            % ulem宏包,提供各种下划线和删除线,normalem选项可以防止\emph{}命令的效果变成加下划线

\usepackage{hologo}
            % hologo宏包可以输出许多TeX家族标志

\usepackage{lettrine}
            % 宏包lettrine能够生成首字下沉的效果

\usepackage{indentfirst}
            %* article文档类下,即使在首行开头使用"\indent"命令,以英文字母开头的首行也不缩进,解决方法有三种:(1)加载indentfirst宏包,这是最常用的办法;(2)加载ctex宏包;(3)设置文档类为ctexart
            %rfr 第二种和第三种方法设置的首行缩进效果会被下文提到的titlesec宏包的"\titlespacing*{command}{left}{before-sep}{after-sep}"命令消除,由此也可以看出加载indentfirst宏包是防止首行不缩进的最好的办法

% \usepackage{parskip}
            %rfr 强制所有段首不缩进,参考:https://tex.stackexchange.com/questions/196922/indent-first-paragraph-after-section-and-dont-indent-new-lines

\usepackage{fontspec} 
            % 在XelaTeX编译下,使用fontspec宏包来选择本地安装的字体。注意,该宏包可能会明显增加编译时间

\usepackage[slantfont,boldfont]{xeCJK} 
            % "slantfont"和"bolfont"选项表示允许设置中文的斜体和粗体字形

    % \setCJKmainfont[boldfont = SimHei]{SimSun} 
            % 设置"SimSun"(宋体)为主要字体,"SimHei"(黑体)为主要字体的粗体字形,即"\textbf{}"或者"{\bfseries}"命令的变换结果。也可以通过slantfont选项来设置主要字体的斜体字行
            %* 各章节的汉字字形也因此发生变化
            %! 注意,"SimSun"和"SimHei"名称中的大小写一定要写对,否则编译会报错

\usepackage{pifont} 
            %* 用于输出图标(symbol fonts)的宏包

\usepackage{xcolor} 
            % 使用xcolor宏包来方便地调用颜色
            % 提供"\colorbox{color}{text}"、"\fcolorbox{color}{color}{text}"命令

\usepackage{amsmath} 
            % 宏包amsmath提供了"\eqref"命令,输出默认形式如"(3.1)"所示的数值

% \usepackage{nameref} 
            % nameref宏包不满足于只引用编号,还提供了引用对象的标题内容的功能
            %* 书中提到"\nameref{label}"命令由nameref宏包提供,但是经检验,即使不加载nameref宏包,这一命令依然生效,说明这一命令应该已经被LaTeX吸收,或者也许本来就是LaTeX的原生命令

\usepackage{lastpage} 
            % lastpage宏包提供的标签"LastPage"可以保证打印出整个文档最后一页的页数

\usepackage[perpage,bottom]{footmisc} 
            % footmisc宏包的perpage选项可以让脚注每页重新编号,默认情况下,脚注按章编号
            %* bottom的可选参数可以使得脚注始终处于默认页面区域的底部,而不会因为LaTeX的提前分页操作导致上移到实际页面区域的底部

\usepackage{endnotes} 
            % 用于设置尾注的宏包

\usepackage{csquotes} 
            %* 管理在正文中援引文本的宏包

\usepackage[numbers,sort&compress,super,square]{natbib} 
            % 使用natbib宏包可以定制参考文献标号在文中的显示方式

\usepackage[numbib,numindex]{tocbibind}
            % 使用tocbibind宏包将目录本身、图目录、表目录、索引、参考文献全部编入目录(以上为默认值,可以通过下述相应参数取消),并且给参考文献章节和索引章节正常编号,可选参数包括:
            % nottoc(目录本身不编入目录)、
            %* "toc"是"table of contents"的缩写
            % notlof(图目录不编入目录)、
            % notlot(表目录不编入目录)、
            % notindex(索引不编入目录)、numindex(给索引章节正常编号,默认情况为不编号)、
            % notbib(参考文献不编入目录)、numbib(给参考文献章节正常编号,默认情况为不编号)、
            % none(禁用所有)
            %! 测试中发现,如果将该宏包放在imakeidx宏包的前面,该宏包不会生效;但是如果将该宏包放在imakeidx宏包的后面,该宏包会生效,但是会导致imakeidx宏包提供的"\makeindex"命令当中的三个可选参数"columns"、"columnsep"、"columnseprule"失效,索引始终保持双栏
            %# 解决的办法是直接使用imakeidx宏包提供的"\indexsetup"命令,添加"level=\section"参数即可

\usepackage[section]{placeins}
            % 重新定义了"\section{}"命令,在此之前加上了"\Float-Barrrier"命令,阻止浮动体跨过该位置

\usepackage{graphicx}
            % 使用graphicx宏包和"\includegraphics[options]{name}"命令插入图片
            % 提供"\scalebox{h-scale}[v-scale]{text}"和"\rotatebox[keyvals]{angle}{text}"命令

\usepackage{caption}
            % caption宏包用来设置浮动体的标题格式

\usepackage{subcaption}
            %* 使用subcaption宏包提供的subfigure环境,在figure环境中嵌套多图
            % 使用subcaption宏包提供的subtable环境,在table环境中嵌套多张表格
            %! 注意,subcaption宏包和subfig宏包只能加载其中一个,否则编译会报错

% \usepackage{subfig}
            % 使用subfig宏包提供的"\subfloat[sub-caption]{body}"命令,在figure环境中嵌套多图
            %! 注意,subcaption宏包和subfig宏包只能加载其中一个,否则编译会报错 
            
\usepackage{wrapfig}
            % 使用wrapfig宏包进行图文混排

\usepackage{multirow}
            % 使用multirow宏包解决跨行跨列的问题

\usepackage{array} 
            % array宏包提供"m{}"、"b{}"、">{decl}"、"<{decl}"、"!{intercolumn sign}"、"\newcolumntype{name}{definition}"等命令

\usepackage{tabularray} 
            %* tabularray宏包提供使文本靠近表格上端或者下端的命令

\usepackage{makecell} 
            % makecell宏包提供了一种方便在单元格内换行的方式,以及设置水平表线宽度的命令

\usepackage{diagbox} 
            % diagbox宏包提供了分割表头的命令

\usepackage{tabularx} 
            % tabularx宏包的tabularx环境使得表格可以通过在每列上均匀增加列宽的方式设置整表宽度

\usepackage{dcolumn} 
            % dcolumn宏包提供了小数点对齐的对齐方式

\usepackage[showframe]{geometry}
    % \geometry{twoside}
            % 借助geometry宏包进行页面设置
            %* twoside选项会影响到边注命令"\marginpar[left]{right}"中可选参数对应的内容在偶数页上的显示位置,如果不设置twoside选项,则偶数页上的边注都一律显示在右侧,参见"3.5.2 脚注、边注与尾注"一节

% \usepackage{fancyhdr}
            % 借助fancyhdr宏包控制页眉和页脚
            %* "hdr"应该是来自"header"的缩写
            %! ctexart文档类提前定义了页眉样式,因此即使不加载fancyhdr宏包,也会显示页眉内容

\usepackage{shortvrb}
            % shortvrb宏包支持以一对符号代替"\verb"命令

% \usepackage{cuted}
            %* cuted宏包提供strip环境,用于解决figure*环境中的图片在双栏模式下无法在首页被插入的问题
            %! 注意,在strip环境中需要使用caption宏包提供的"\captionof{float type}{heading}"命令来设置标题

% \usepackage{dblfloatfix}
            %* dblfloatfix宏包用于解决figure*环境中的图片在双栏模式下无法在底部被插入的问题,但是仍然无法解决图片无法在首页被插入的问题
            %! 注意,dblfloatfix宏包和stfloats宏包不兼容,只能加载其中一个
            %* 但是dblfloatfix宏包可以和nidanfloat宏包兼容

% \usepackage{stfloats} 
            %* stfloats宏包可以同时解决上述两个问题

% \usepackage{nidanfloat} 
            %* nidanfloat宏包也可以同时解决上述两个问题
            %! 注意,加载nidanfloat宏包会导致通过cuted宏包提供的strip环境中的图片消失,但编译不会报错

\usepackage{multicol}
            % multicol宏包提供multicols环境,可以使同一页内分栏与单栏并存,或者分成多栏,但是此时将无法使用不带星号的浮动体和边注
            %! 注意不要将multicols环境和表格中用于合并列的"\multicolumn{n}{cols}{text}"混淆

\usepackage{import} 
            %* import宏包提供了"\import{full path}{file}"命令和"\subimport{relative path}{file}"命令,前者类似"\input{}"命令,后者可以嵌套使用:在文件一中使用"\import{full path}{file}"命令引用文件二,在文件二中使用"\subimport{relative path}{file}"命令引用文件三,这样就可以直接在文件一中使用"\import{full path}{file}"命令引用文件三

\usepackage{boxedminipage}
            % boxedminipage宏包提供加框小页环境
            
\usepackage{changepage}
            % changepage宏包提供了一个adjustwidth环境,能够控制段落两侧到文本区的距离

\usepackage{titlesec}
            % 通过titlesec宏包自定义正文中的章节样式

\usepackage{titletoc}
            % 通过titletoc宏包自定义目录中的章节样式

\usepackage{longtable}
            % longtable宏包用来排版长表格

\usepackage{booktabs}
            % booktabs宏包提供"\toprule"、"\midrule"与"\bottomrule"命令来绘制三线表

\usepackage{colortbl}
            % 彩色表格依靠colortbl宏包,它会调用array宏包和color宏包,可以在加载xcolor宏包时添加table选项来调用colortbl宏包

\usepackage[shortlabels,inline]{enumitem}
            % enumitem宏包允许在列表环境开始的时候添加一个可选参数,在可选参数中通过"<key>=<value>"的形式来控制列表的相关参数,也可以通过"\setlist[]{}"命令来为各级列表设置统一的参数
            %rfr 可以参考enumitem宏包的官方手册,尽管这一手册非常难读:https://sg.mirrors.cicku.me/ctan/macros/latex/contrib/enumitem/enumitem.pdf

\usepackage{makeidx}
            % 使用makeidx宏包来建立索引

\usepackage{showidx}
            %* showidx宏包用来将设置索引的命令内容显示在当前页面的边注位置,通常在撰写文档过程中使用,方便立刻定位到所设置的索引位置
            %! 注意,这一宏包需要搭配hyperref宏包使用,否则编译会报错

\usepackage{imakeidx} 
            %* imakeidx宏包重定义了"\makeindex"命令,允许为其添加可选参数来设置索引样式(比如将索引加入目录、设置索引分栏等、为索引分组等)
            %# 经检验,单独加载该宏包也可以建立索引

\usepackage[titletoc]{appendix}
            % 使用appendix宏包设置附录样式
            %* 这需要使用该宏包提供的appendices环境而不是LaTeX的原生命令"\appendix"
            % 可选参数"titletoc"表示在目录中附录计数器(section/chapter)的添加一个名称,默认值为"Appendix"
            %* 在ctexart文档类中,则显示为"附录A"
    \renewcommand{\appendixname}{这是附录}
            % 重定义"\appendixname"命令来修改目录当中附录计数器前面的名称

\usepackage[switch,pagewise,modulo]{lineno}
            % 使用lineno宏包可以生成行号
            
\usepackage{hyperref}
            % 索引更常用的是hyperref宏包,由于它经常与其他宏包冲突,一般把它放在导言区的最后
            % 加载hyperref宏包后,目录可以实现点击跳转
    \hypersetup{colorlinks = true,linkcolor=blue,anchorcolor=blue,citecolor=blue,urlcolor=red}
            % hyperref宏包的选项也可以以"\hypersetup"的形式另起一行书写,colorlinks选项中的默认值是:colorlinks = false,linkcolor = red, anchorcolor = black, citecolor = green, urlcolor = magenta
            %* 如果以上命令作为hyperref宏包的可选参数,应该将其置于必选参数"{hyperref}"的前面,否则设置的可选参数无法生效


% --------------------------------------------------------------------------------------------------------------------------------------