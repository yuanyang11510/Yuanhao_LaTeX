\documentclass[twoside]{article}
\usepackage{ctex}
\usepackage[showframe]{geometry}
\usepackage{fancyhdr}

% 设置正文的页面样式
\pagestyle{fancy}

% 为索引部分设置单独的页面样式,在"\indexsetup"命令的"firstpagestyle"参数中使用
%rfr 记得将页眉基线和页脚基线的宽度设置为0pt,参考:https://tex.stackexchange.com/questions/13896/how-to-remove-the-top-horizontal-bar-in-fancyhdr
\fancypagestyle{idxstyle}{
    \renewcommand{\headrulewidth}{0pt}
    \renewcommand{\footrulewidth}{0pt}
    \fancyhf{}
    \fancyhead[c]{\leftmark}
    \fancyfoot[c]{\thepage}
}

\usepackage[numindex]{tocbibind}
    %* 如果还要让索引正常编号,需要加载tocbibind宏包并且设置numindex参数,当然,这一宏包的参数本身就能同时实现索引正常编号和加入目录的功能
    %! 测试中发现,如果将该宏包放在imakeidx宏包的前面,该宏包无法使索引正常编号(但是其他功能正常);如果将该宏包放在imakeidx宏包的后面,该宏包会生效,但是会导致imakeidx宏包提供的"\makeindex"命令当中的三个可选参数"columns"、"columnsep"、"columnseprule"失效,索引始终保持双栏
    %# 解决的办法是:(1)将tocbibind宏包放在imakeidx宏包的前面,防止其影响"columns"、"columnsep"、"columnseprule"三个参数;(2)直接使用imakeidx宏包提供的"\indexsetup"命令,添加"level=\section"参数即可,见下文的详细介绍

% \usepackage{makeidx}
    % 使用makeidx宏包来建立索引
    %rfr makeidx宏包的相关内容大多参考了这一网站的Index部分:https://latexref.xyz/,也可以参考:https://texdoc.org/serve/lshort-zh-cn.pdf/0
    %* 生成索引部分需要三步:(1)在第一遍编译时,"\makeindex"命令会将文档当中所有的索引信息保存到一个.idx文件当中;(2)接下来在终端中使用"makeindex+.idx文件名"命令(可以选择带或者不带扩展名)生成一个.ind文件;(3)再次编译,就可以在"\printindex"命令处生成索引的条目陈列部分
    %rfr 在生成索引之后,除非对索引命令"\index{entry}"当中的参数进行了修改,否则如果调整了其他参数只需要重新编译即可,不需要重复以上三步,可以参考:https://tex.stackexchange.com/questions/530890/indexsetup-and-intoc-from-imakeidx-package-problem

\usepackage{showidx}
    %* showidx宏包用来将设置索引的命令内容显示在当前页面的边注位置,通常在撰写文档过程中使用,方便立刻定位到所设置的索引位置
    %! 注意,这一宏包需要搭配hyperref宏包使用,否则编译会报错
\usepackage[colorlinks,linkcolor=blue]{hyperref}
    
\usepackage{imakeidx}
    %* imakeidx宏包重定义了"\makeindex"命令,允许为其添加可选参数来设置索引样式(比如将索引加入目录、设置索引分栏、为索引分组等)
    %# 经检验,单独加载该宏包也可以建立索引
    %rfr imakeidx宏包的相关内容参考了:https://www.overleaf.com/learn/latex/Indices,也可以参考其官方手册:https://ftp.yz.yamagata-u.ac.jp/pub/CTAN/macros/latex/contrib/imakeidx/imakeidx.pdf
    %! 练习时发现,在普通的article文档类当中,最后索引条目会在默认双栏范围内排列,但是排列的效果非常奇怪,并不会先填满第一栏然后填充第二栏,而是在两栏中均匀分布
    %rfr 为了能够使索引条目能够逐栏填充,可以设置"original"的宏包参数,可以参考:https://tex.stackexchange.com/questions/268018/imakeidx-columns-splitting
    %# 但是如果加载了ctex宏包,就不会存在这一问题

% 以下"\makeindex"命令的可选参数都来自imakeidx宏包---------------------------------------------------------------------------------------------------------------------------------------------------------------------------------------
\makeindex[columns=3,columnsep=50pt,columnseprule]
\makeindex[title={Group 1},name=g1]
\makeindex[title={Group 2},name=g2]
% "intoc"参数表示将相应的索引部分加入目录
%rfr 如果在"\indexsetup"命令中为"level"参数设置了不带"*"符号的章节命令,则这一参数不需要设置,见下文介绍
%* "columns"参数用来设置索引的栏数(官方手册没有明确指出其默认值,但是不管是使用makeidx宏包还是使用没有设置columns参数的imakeidx宏包,最后的索引部分都是双栏,可知该参数的默认值应该是2);"columnsep"参数用来设置栏间距(默认值为35pt);"columnseprule"参数用来显示分隔线
%! 注意,imakeidx宏包的官方手册提到,如果为宏包设置了"original"参数,或者将文档设置为了双栏模式,上述三个参数都不会生效,前文也已经提到,如果将tocbibind宏包放在imakeidx宏包后面,也会导致这三个参数失效
%* 每设置一个"\makeidx"命令并且设置了"name"参数(之后作为"\index{entry}"命令的可选参数使用),就意味着多一个索引分组("title"参数反而不是必须的,如果不设置则默认为下文介绍的"\indexname"命令的值),相应的,在最后也要使用带有相应可选参数的"\printindex"命令才能将分组后的索引打印出来

% 以下"\indexsetup"命令来自imakeidx宏包---------------------------------------------------------------------------------------------------------------------------------------------------------------------------------------
\indexsetup{
    level=\section,%
% "level"参数表示为每个索引标题使用相应的章节命令,章节命令不需要带参数
%* 如果章节命令设置为带"*"号的版本,则不会加入目录(这其实是默认情况,因此没有必要进行这一步设置);如果设置为不带"*"号的版本,则会加入目录,这就达到了相当于tocbibind宏包提供的索引正常编号功能
%rfr 此时就不需要在"\makeindex"命令中添加"intoc"参数,否则会导致在目录出现两个索引标题:一个不带编号(来自"intoc"参数),一个带编号(来自"level"参数),可以参考:https://tex.stackexchange.com/questions/530890/indexsetup-and-intoc-from-imakeidx-package-problem
    % firstpagestyle=plain,%
    % firstpagestyle=empty,%
    % firstpagestyle=headings,%
    % firstpagestyle=fancy,%
    firstpagestyle=idxstyle,%
%* "firstpagestyle"参数设置索引部分的页面格式(页眉、页脚),可以设置的值除了和fancyhdr宏包当中相同的plain、empty、headings之外,也可以设置自己定义的页面格式
%! 这一参数的名字具有误导性,其控制的是索引部分每一页上的页面格式,不止第一页
%* imakeidx宏包的官方手册指出,该参数会因为加载fancyhdr宏包而失效,事实上情况比较复杂:在加载了fancyhdr宏包的情况下,此处设置为"plain"、"empty"都可以生效,设置为"headings"时,不会完全生效:所有对应"rightmark"的页眉会消失,但是"headings"在此处本来也不是一个很好的选择(见下文介绍),因此,完全可以加载fancyhdr宏包,为正文和索引各自独立地设置页面格式:这是因为在正文当中通过"\pagestyle{}"命令设置的页面格式并不会影响到索引,而索引部分的页面格式可以通过此处的"firstpagestyle"参数进行设置
%# 此处我们就使用了一个在导言区专门为索引部分定义的页面格式,通过最终的打印效果我们可以发现,在自定义页面格式的情况下,每个索引部分的标题都被识别为了"leftmark",而这个和前文的"level=\section"参数没有关系,经检验,即使不设置"level"参数,此处的效果保持不变
    headers={\Huge\indexname}{\small\indexname},%
%* "headers"参数在设置了上一个参数的基础上,设置页眉的字体样式,前后两个必选参数分别控制"leftmark"参数(即“较高级别的信息”)和"rightmark参数"(即“较低级别的信息”),两个必选参数缺一不可,即使在有些情况中只有其中一个会发挥作用
%! 如果"firstpagestyle"参数设置为"headings",在单边模式下,系统会将所有的索引标题识别为"rightmark";而在双边模式下,情况会变得非常奇怪:系统会将奇数页上的索引标题识别为"rightmark",将偶数页上的索引标题识别为"leftmark",因此,在双边模式下,将"firstpagestyle"参数设置为"headings"并不是一个很好的选择,即使设置了,也最好不要为两个参数设置不同的字体样式
    % noclearpage,%
%* "noclearpage"参数表示在每一个索引部分之前不使用"\clearpage"命令,即让所有的索引部分接着前面的内容打印而不分页打印
%! 但是这一设置可能会导致索引部分的页眉显示异常,比如在这篇文档当中,如果在正文和索引之间插入一条"\clearpage"命令,再启用此处的"noclearpage"参数,最后打印出来的索引部分的最后一页会同时出现正文和索引部分的两种页面格式
}

% 通过重定义"\indexname"命令来更改索引标题---------------------------------------------------------------------------------------------------------------------------------------------------------------------------------------
\renewcommand{\indexname}{这是索引}

% ---------------------------------------------------------------------------------------------------------------------------------------------------------------------------------------

\begin{document}
\tableofcontents

\section{第一页}
% 最基本的索引设置命令"\index{entry}"
    A\index{A}

% 在"\index{entry}"命令当中使用"!"符号设置索引次级条目和次次级条目,最多可以设置三级
    B\index{B}
    B'\index{B!B'}
    B''\index{B!B'!B''}
    %*// 在最后的索引条目陈列部分,只有最后一级的条目会显示页码

% 在"\index{entry}"命令当中使用"@"符号为索引条目设置另外的形式
    C\index{C@c}
    x-D\index{D@\textbf{x-D}}
    
    %* 如果没有"@"符号,"\index{entry}"命令的参数就是最后的索引条目的呈现形式,使用"@"符号之后,最后的索引条目形式由"@"符号后面的内容控制,"@"符号前面的内容则用于排序
    %# 这提醒我们,最后的索引条目并不是简单地按照在正文中出现的先后顺序来排序的,而是重新按照首字母的先后顺序排序(经检验,特殊字符出现在字母之前)
    %! 在测试中发现,如果使用"@"符号设置了不同的索引条目样式,但是共用同一个“排序字符”,使用makeindex处理.idx文件时不会产生警告,实际上,这是一种很常见的操作,即为正文中的几个相近形式(尤其是那种起首字符不相同但是之后的某些部分相同的情况)的内容各自设置索引,并且让最后的几条相应的索引条目排列在一起

% 在"\index{entry}"命令当中使用"|"符号设置索引条目的页码样式
% 一种特殊的用法是使用"|("和"|)"来输出在连续多页上出现的索引条目的页码
    E\index{E|(}

\newpage

\section{第二页}
    E\index{E|)}
    y-D\index{D@\textit{y-D}}

% 注意在"|"符号之后的命令【不能】写出开头的"\"符号,并且如果只含有索引条目页码一个参数,这一参数也不再写出
    F\index{F|textbf}
    %// F\index{F|textit}
    %* 经检验,如果在"|"符号之后的命令把开头的"\"符号写出来,编译不会报错,但是这一命令会被识别为文本添加到页码的前面,并且页码会另起一行在稍后的位置打印出来,效果非常奇怪
    %* 经检验,即使是不需要带参数的声明型字体命令(比如"\bfseries")也可以放在"|"符号后面,但是这会影响到出现在该条目之后的所有条目的页码,因此不推荐使用
    %! 在测试过程中发现,如果在同一页上使用"|"符号为同一个索引条目设置了不同样式的的页码(包括下文的"\see{}{}"命令和"\seealso{}{}"命令),在使用makeindex处理.idx文件时会产生警告,但是不影响编译,最后会将不同样式的几个页码都打印出来按照先后次序排列在一起

%* 在"|"符号之后只能输入一条命令,如果想要对索引条目页码应用多条命令(比如同时加粗和加斜),需要先自定义一个包含这些命令效果的新命令,再将新命令放在"|"符号之后
    \newcommand{\bfit}[1]{\textbf{\textit{#1}}}
    %// \newcommand{\bfit}[1]{\bfseries\itshape}
    G\index{G|bfit}
    %! 注意要定义为一个参数型命令,否则不仅会导致其影响该条目之后的其他条目的页码,而且其自身只会打印出条目,但是不打印页码

% makeidx宏包提供的"\see{}{}"命令和"\seealso{}{}"命令经常用在"\index{entry}"命令当中,这两条命令除了索引条目页码,还含有另一个参数用于交叉引用,因此在"|"符号后面需要写出另外一个参数
    H\index{H|see{A}}
    I\index{I|seealso{B}}

% 还可以将上述的"!"、"@"和"|"符号搭配使用
    J\index{J!J'@j'|textit} % 意为增加一个次级索引,并且为这个次级索引条目设置了另外的条目样式和页码样式
    K\index{K@k!k'|textbf} % 意为为一个索引设置了另外的条目样式,然后增加一个次级索引,并且为这个次级索引条目设置了页码样式
    %! 经检验,"|"符号需要放在最后使用,否则编译虽然不会报错, 但是最后不会打印出对应的索引

% 如果要在最后的索引条目或者页码中输出上述三个拥有特定功能的字符"!"、"@"、"|",需要在其前面加上一个双引号,如果要输出双引号本身,需要将双引号连写
    L\index{"!L}
    M\index{"@M}
    N\index{"|N}
    O\index{""O}
    %* 但是同样拥有特定功能的"("和")"并没有这一要求,这可能是因为这两个符号发挥特定功能的环境特别受限(只能是"|("和"|)"),而另外三个符号使用的环境则要自由很多

\newpage

\section{第三页}
% 通过imakeidx宏包为"\makeindex"命令设置可选参数"name",在正文中作为"\index{entry}"命令的可选参数使用,作为索引分组的依据
    P\index[g1]{P}
    Q\index[g2]{Q}

\newpage

\section{第四页}
    R\index{R}
    S\index{S}
    T\index{T}
    U\index{U}
    V\index{V}
    W\index{W}
    X\index{X}
    Y\index{Y}
    Z\index{Z}
    a\index{a}
    b\index{b}
    c\index{c}
    d\index{d}
    e\index{e}
    f\index{f}
    g\index{g}
    h\index{h}
    i\index{i}
    j\index{j}
    k\index{k}
    l\index{l}
    m\index{m}
    n\index{n}
    o\index{o}
    p\index{p}
    q\index{q}
    r\index{r}
    s\index{s}
    t\index{t}
    u\index{u}
    v\index{v}
    w\index{w}
    x\index{x}
    y\index{y}
    z\index{z}

% \clearpage

\printindex
\printindex[g1]
\printindex[g2]
% 导言区设置了几个"\makeindex"命令,最后就要使用几个相应的"\printindex"命令,将分组后的索引全部打印出来

\end{document}