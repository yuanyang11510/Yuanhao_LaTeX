\documentclass{article}
\usepackage[scheme = plain]{ctex}
% \usepackage{parskip} % 取消首行缩进
\usepackage[showframe]{geometry}
\usepackage[english]{babel}
\usepackage{lipsum}
\usepackage{blindtext}
\usepackage{amsthm}
\usepackage[framemethod = tikz]{mdframed}
\usetikzlibrary{shadows}

% mdframed宏包自带框环境参数设置
\mdfsetup{
    frametitle = {Frame-Title\\ABC},
    frametitlefont = \bfseries\itshape,
    frametitlerule = true,
    frametitlebackgroundcolor = orange!30,
    % v
    backgroundcolor = yellow!30,
    fontcolor = violet,
    % 
    topline = false,bottomline = false,rightline = false,
    innerlinewidth = 2pt,innerlinecolor = red,
    middlelinewidth = 2pt,middlelinecolor = blue,
    outerlinewidth = 2pt,outerlinecolor = green,
    % 
    roundcorner = 2pt,
    shadow = true
}

\begin{document}

mdframed宏包自带框环境
\begin{mdframed}
    foo foo foo
\end{mdframed}

副标题
\begin{mdframed}
    \mdfsubtitle[
        subtitlefont = \itshape,
        subtitlebackgroundcolor = red!30,
        subtitleaboveskip = 0pt,
        subtitlebelowskip = .5\baselineskip
    ]{Frame-Subtitle}
    foo foo foo
\end{mdframed}

% ------------------------------------------------------------------------------------------------------------------------------------------------------------------

自定义框环境风格
\mdfdefinestyle{mystyle}{leftmargin = 1cm,linecolor=blue}
\begin{mdframed}[style = mystyle]
    foo foo foo
\end{mdframed}

环境可选参数设置default,忽略mdfsetup命令中设置的所有参数
\begin{mdframed}[default,style = mystyle]
    foo foo foo
\end{mdframed}

% ------------------------------------------------------------------------------------------------------------------------------------------------------------------

第一种自定义框环境,如果不设置框标题,默认为环境名称
\newmdenv[linecolor = red,frametitle = XYZ]{infobox}
\begin{infobox}[backgroundcolor = yellow]
    foo foo foo
\end{infobox}

环境可选参数设置default,效果同上
\begin{infobox}[default,backgroundcolor = yellow]
    foo foo foo
\end{infobox}

% ------------------------------------------------------------------------------------------------------------------------------------------------------------------

第二种自定义定理环境,框标题默认为环境名称,即使另外设置也无法改变,框名称自带计数器
\mdtheorem[linecolor = red]{xyz}{XYZ}
\begin{xyz}
    foo foo foo
\end{xyz}

% ------------------------------------------------------------------------------------------------------------------------------------------------------------------

mdframed宏包自定义定理环境,默认适用框环境
\theoremstyle{remark}
\newmdtheoremenv[linecolor = blue]{remark}{Remarque}[section]
\begin{remark}[Some title]
    foo foo foo
\end{remark}

% ------------------------------------------------------------------------------------------------------------------------------------------------------------------


\LaTeX{}自带定义定理环境,不适用框环境
\newtheorem{opq}{OPQ}
\begin{opq}
    foo foo foo
\end{opq}

通过mdframed宏包提供的surroundwithmdframed命令适用框环境
\surroundwithmdframed{opq}
\begin{opq}
    foo foo foo
\end{opq}

% ------------------------------------------------------------------------------------------------------------------------------------------------------------------



\end{document}