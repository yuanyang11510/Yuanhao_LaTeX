\documentclass{article}
% \usepackage{colortbl}
\usepackage{longtable} % longtable宏包用来排版长表格
%rfr 下面尝试着对着longtable宏包官方手册中给出的第一个长表格的例子制作一个相同的表格,可以参考:https://sg.mirrors.cicku.me/ctan/macros/latex/required/tools/longtable.pdf
\usepackage[bottom]{footmisc} % footnotemis宏包搭配bottom可选参数防止页面的下边线因为提前分页而升高

    % longtable环境默认居中对齐,即左右两侧到文本左右边线的距离默认为\fill,可以通过以下命令来使其左对齐
    % \setlength{\LTleft}{0pt}
    %* 如果将"\LTleft"改为"\LTright",则为右对齐,当然,也可以设置其他距离;如果将"\LTleft"改为"\LTpre",则可以控制表格上方到前面正文的距离,默认值为\bigskipamount,通常是一行左右的竖直距离,即12pt +/- 4pt左右;如果将"\LTleft"改为"\LTpost",则可以控制表格下方到后面正文的距离

\begin{document}
\begin{longtable}{@{*}r||p{1in}@{*}}
    KILLED&LINE!\kill 
    % "\kill"表示该行不显示,但用于计算宽度
    %rfr 何为“计算宽度”?可以参照"5.3.4 制表位*"一节当中的用法,此处可以将"KILLED"、"LINE!"这两格内容替换成距离命令"\hspace{l}",这可以替代longtable环境必选参数当中的"p{}"参数,用来设定表格宽度。"p"是"parbox"的首字母,可以参考:https://tex.stackexchange.com/questions/35293/p-m-and-b-columns-in-tables
    %* 注意,如果在单元格的末尾使用距离命令,需要加上"*"符号,否则无法添加指定距离的空格
    \caption[An optional table caption (used in the list of tables)]{A long table}\\
    %! longtable环境当中的"\caption{}"命令最后需要添加"\\"命令,否则编译会报错
    \hline\hline
    %! 如果同时加载colortbl宏包和longtable宏包,会导致在longtable环境中,连续两个"\hline"命令绘制的两根水平表线之间的距离会大幅减小(消失?),看起来就像是绘制了一根更粗的水平表线,而不是像tabular环境中绘制两条平行的水平表线
    %* 有的时候,如果表头/表尾当中设置的水平表线碰巧和表体当中设置的水平表线重叠,这种情况下,即使不加载colortbl宏包,也会导致叠加后的水平表线变粗
    \multicolumn{2}{@{*}c@{*}}{This part appears at the top of the table}\\
    FIRST&SECOND\\
    \hline\hline
    \endfirsthead
    % 定义首页表头(如果首页表头与其它页不同的话)

    \caption[]{(continued)}\\
    %* longtable环境重定义了"\caption[]{}"命令,原命令中的可选参数表示在表格目录中显示的标题,如果留空,则表格目录中的对应标题会留白,但是该目录项不会消失,新定义后的命令如果将可选参数留空,可以不将该表格项加入表格目录,这一用法通常应用于首页之后的各页表格标题(因为只有首页的表格标题才需要加入目录)
    %* longtable环境还新定义了"\caption*{text}"命令,打印出的标题只会显示参数内容,而不会在这之前显示"Table X: "的字样,即不会进行编号,并且该标题也不会被加入表格目录
    \hline\hline
    \multicolumn{2}{@{*}c@{*}}{This part appears at the top of every other page}\\
    \textbf{First}&\textbf{Second}\\
    \hline\hline
    \endhead
    % 定义首页之后的各页表头

    \hline
    This goes at the&bottom\\
    \hline
    \endfoot
    % 定义末页之前的各页表尾

    \hline
    These lines will&appear\\
    in place of the&usual foot\\
    at the end&of the table\\
    \hline
    \endlastfoot
    % 定义末页表尾(如果末页表尾和其它页不同的话)
    %* 以上定义各页表头、表尾的命令不是必须的,即使不定义也能完成编译
    %! 书中和longtable宏包官方手册均指出,"\label{}"命令不能用在重复出现的对象当中,因此只能用于首页表头、末页表尾和表体当中,不能用于首页之后的各页表头和末页之前的各页表尾,但是经检验这并不影响编译,点击对应的"\ref{}"命令也总是索引到同一位置:longtable的第一页

    % 以下是长表格的表体部分
    % \hline
    \textsf{longtable} columns are specified&in the\\
    same way as in the \textsf{tabular}&environment.\\
    \textsf{@\{*\}r||p\{1in\}@\{*\}}&in this case.\\
    Each row ends with a&\textsf{\textbackslash\textbackslash} command.\\
    The \textsf{\textbackslash\textbackslash} command has an&optional\\
    argument, just as in&the\\
    \textsf{tabualr}&environment.\\[10pt]
    % "\\"命令后可以设置可选参数插入竖直空距
    % 如果在"\\"命令后加上"*"符号,等同于加上"\nopagebreak"命令,表示在此处禁止分页
    %* 经检验,这一设置只适用于表体当中,不适用于表头、表尾和表体的最后一行
    See the effect of \textsf{\textbackslash\textbackslash[10 pt]}&?\\
    Lots of lines&like this.\\
    Lots of lines&like this.\\
    Lots of lines&like this.\\
    Lots of lines&like this.\\
    Also \textsf{\textbackslash{}hline} may be used,&as in \textsf{tabular}.\\
    \hline
    That was a \textsf{\textbackslash{}hline}&.\\
    \hline\hline
    That was \textsf{\textbackslash{}hline\textbackslash{}hline}&.\\
    \multicolumn{2}{||c||}{This is a \textsf{\textbackslash{}multicolumn\{2\}\{||c||\}}}\\
    If a page break occurs at a \textsf{\textbackslash{}hline} then&a line is drawn\\
    at the bottom of one page and at the&top of the next.\\
    \hline
    The \textsf{[t] [b] [c]} argument of \textsf{tabular}&can not be used.\\
    The optional argument may be one of&\textsf{[l] [r] [c]}\\
    to specify whether the table should be&adjusted\\
    to the left, right&or centrally.\\
    \hline\hline
    Lots of lines&like this.\\
    Lots of lines&like this.\\
    Lots of lines&like this.\\
    Lots of lines&like this.\\
    Lots of lines&like this.\\
    Lots of lines&like this.\\
    Lots of lines&like this.\\
    Lots of lines&like this.\\
    Lots of lines&like this.\\
    Lots of lines&like this.\\
    Lots of lines&like this.\\
    Lots of lines&like this.\\
    Lots of lines&like this.\\
    Lots of lines&like this.\\
    Lots of lines&like this.\\
    Lots of lines&like this.\\
    Lots of lines&like this.\\
    Lots of lines&like this.\\
    Lots of lines&like this.\\
    Lots of lines&like this.\\
    Some lines may take up a lot of space, like this:&This last column is a ``p'' column so this ``row'' of the table can take up several lines. Note however that \TeX\ will never break a page within such a row. Page breaks only occur between rows of the table or at \textsf{\textbackslash{}hline} commands.\\
    Lots of lines&like this.\\
    Lots of lines&like this.\\
    Lots of lines&like this.\\
    Lots of lines&like this.\\
    Lots of lines&like this.\\
    Lots of lines&like this.\\
    Lots of lines&like this.\\
    \hline
    Lots of lines\footnote{This is a footnote.}&like this.\\
    Lots of lines&like this\footnotemark\footnotetext{\textsf{longtable} takes special precautions, so that footnotes may also be used in `p' columns.}.\\
    % \hline
    %* 在longtable环境使用脚注的两种方式和在普通的table环境中不同:
        %* (1)在longtable环境的表体中可以直接使用"\footnote{}"命令,而在table环境中虽然可以使用,但是不会打印出脚注;
        %* (2)在longtable环境的表体中可以搭配使用"\footnotemark"命令和"\footnotetext{text}"命令,由于后者决定了脚注打印的位置,因此需要放置在和脚注位置打印在同一页面的表体的任何位置(一般紧跟在"\footnotemark"命令之后),如果放在表头或者表尾中,脚注不会打印出来,而在table环境中虽然也可以搭配使用这两条命令,但是"\footnotetext{text}"命令必须放在环境结束之后,否则不会打印出脚注(由于table环境打印出的表格不会跨页,因此不存在longtable环境当中【选择"\footnotetext{text}"命令的插入位置】的问题)。这里还涉及"\footnotetext{text}"命令的编号问题:在longtbale环境中,"\footnotetext{text}"命令的编号会根据排列的先后次序自动编号,也可以通过可选参数指定,而在table环境中,由于所有的"\footnotetext{text}"命令都必须放置在环境结束后,因此当存在多个该命令时,就必须通过可选参数为这些命令指定编号,或者通过\addtocounter{}{}命令和\stepcounter{}命令为\footnotetext{text}命令重新调整编号,否则这些脚注的编号都会等于最后一条"\footnotemark"命令对应的编号;
        %* (3)在longtable环境的表头、表尾和"\caption{}"命令中,不能直接使用"\footnote{}"命令,否则不会打印出脚注,搭配使用"\footnotemark"命令和"\footnotetext{text}"命令时,需要将其放置在和脚注位置打印在同一页面的表体的任何位置,可以发现,在longtable环境中,在"\caption{}"命令中使用"\footnotemark"命令不再需要在前面加上"\protect"命令,而在table环境中,这是必须的,否则编译会报错
    \hline
    Lots of lines&like this.\\
    Lots of lines&like this.\\
    % \hline
\end{longtable}

    % \newpage
    %* 最后还涉及到LaTeX提前分页的问题,通过打印效果可以发现,最后一页上的脚注紧贴着longtable的底部,这其实反映出LaTeX自动在longtable环境结束后添加了一条"\pagebreak"命令,使得该页提前结束了,因此脚注会跟着跑到分页的位置,造成了当前页面底部出现大片空白(underfull)的情况,经检验,这一情况在另外的tex文件中单独打印该长表格时没有出现,因此这一情况应该是LaTeX自身在同一文件中存在其他内容时,自动排版导致的
    %! 以上所述“提前分页”是原来在文件"LaTeX_Wu_Kanglong.tex"中编译时出现的问题
    %rfr 关于提前分页的问题,可以参考这一帖子:https://tex.stackexchange.com/questions/4902/why-is-my-footnote-glued-to-the-text,其中涉及几个要点:(1)LaTeX中很重要的一个概念"penalty",简单来说,如果penalty的值小于等于-10000,则强制换页,如果大于等于10000,则禁止换页,如果处于-10000和10000之间,则体现为换页或者不换页的倾向;(2)排版当中换页的两种情况:第一种情况最常见,那就是一个段落在当前页面无法全部填充,则只能换页继续填充,这会导致一个段落被分割成两个甚至多个部分,分布在多个页面上;第二种情况比较特殊,那就是前一个段落的最后一行刚好位于当前页面的底部,下一个段落只能换页填充,这就导致前一个段落的最后一行在前一个页面的底部,后一个段落的第一行在后一个页面的顶部。LaTeX中的penalty值影响到的是前一种情况:以这篇帖子为例,将相邻两行之间的penalty值设为了10000,这似乎说明永远都不允许换页,因为前后两行总是连在一起的,但实际情况却是:当前后两段能够放置在同一页面中时,不换页,一旦两段无法同时放置在同一页面中时,这两段会被分别放置在不同的页面中,此时如果增加任何一段的长度,这一段的内容都会不断往下延伸,最终超出页面范围而不会换页,这说明LaTeX此时仅仅是禁止同一段落内禁止换页,当同一页面无法容纳多个段落时,则允许段落之间进行换页;(3)寡行(widow)和孤行(orphan),这是排版当中的两个重要术语,寡行就是一个段落的最后一行出现在了后一页上,和前一页上的各行分开,孤行就是一个段落的第一行出现在了前一页上,和后一页上的各行分开,LaTeX的提前分页有时就是为了避免寡行或者孤行
    %rfr 也可以参考这一篇帖子当中提供的例子:https://www.zhihu.com/question/462017394,第一个例子提供了提前分页的另一种情况:后一段由于处于环境之中,因此无法在内部进行分页,导致LaTeX只好提前换页,后一个例子则是直接使用"\pagebreak"强行制造换页,这也体现出"\pagebreak"和"\newpage"、"\clearpage"等命令的不同之处,前者直接换页,这会导致实际的页面高度小于默认值,而脚注的位置永远是由页面底部位置决定的,因此此时脚注就会往上移动,看起来好像是紧贴着正文一样,后者则会在换页之前插入一个"\vfil"命令,以维持实际的页面高度等于默认值,因此脚注将始终位于页面底部,其实这两种情况下,脚注都位于页面底部,只不过前一种情况中实际的页面底部发生了上移,我们可以将四边发生移动后的区域称为“实际页面区域”,而将表面上未曾移动的区域称为“默认页面区域”
    %* 我们没有必要去完全弄明白提前分页产生的原因(回看本文档前面的内容会发现还存在其他的提前分页的情况),如果想要让页脚始终处于默认页面区域的底部,只需要加载footmisc宏包并且设置bottom的可选参数即可,但这不会使得导致提前分页的"\pagebreak"消失,另一种临时的解决办法是插入"\newpage"命令,但是这样设置就无法让后续内容在当前页面继续打印,而必须要另起一页打印

    \end{document}