\documentclass{article}
\usepackage{ctex}
\usepackage[showframe]{geometry}

\begin{document}
%rfr 可以参考该网站(以下简称“网站”)对"\llap{text}"命令和"\rlap{text}"命令的介绍,但是介绍的比较简略:https://jp.mirrors.cicku.me/ctan/info/impatient/cn/cnbook.pdf,受这篇帖子(以下简称“帖子”)的启发:https://tex.stackexchange.com/questions/146098/llap-or-rlap-at-the-beginning-of-an-indented-paragraph,这两条命令的用法可以总结为以下四条命令:(1)"a\llap{b}"、(2)"\indent\llap{a}b"、(3)"\indent\rlap{a}b"、(4)"a\rlap{b}",其实它们都可以通过"\makebox[width][position]{text}"命令达到相同的效果
    a\llap{b}\dotfill a\llap{b}
    %* 根据网站的说法,参数文本的初始打印位置会从命令前文本的右端向左移动自身长度的距离,最终参数文本的右端和命令前文本的右端重合

    a\makebox[0pt][r]{b}\dotfill a\makebox[0pt][r]{b}
    %# 受帖子的启发,我们可以很清楚地发现,网站的说法实际上等同于在命令前文本的右边创建一个宽度为0pt的箱子,并且设置箱子内文本为右对齐

\hrulefill

    a\rlap{b}\dotfill a\rlap{b}
    %* 关于这种情况,网站没有做任何说明

    a\makebox[0pt][l]{b}\dotfill a\makebox[0pt][l]{b}
    %# 受帖子的启发,我们可以很清楚地发现,这种情况表面上看起来是等同于直接打印相应的文本内容,实际上并非如此,其相当于在命令前文本的右边创建一个宽度为0pt的箱子,并且设置箱子内文本为左对齐

    ab\dotfill ab
    %# 这里同时列出直接打印相应文本内容的命令,可以发现,在文本区域右边线处,可以体现出其和上面两条命令的不同:由于"b"占据了一个箱子的宽度,因此并不会超出文本区域打印,这和前两条命令的性质是完全不同的

\hrulefill

    \indent\llap{a}b\dotfill \indent\llap{a}b
    %* 根据网站的说法,参数文本的初始打印位置会从缩进2em的地方向左移动自身长度的距离,最终参数文本的右端和命令后文本的左端重合,两者没有相交部分

    \makebox[0pt][r]{a}b\dotfill \makebox[0pt][r]{a}b
    %# 受帖子的启发,我们可以很清楚地发现,网站的说法实际等同于在命令后文本的左边创建一个宽度为0pt的箱子,并且设置箱子内文本为右对齐

\hrulefill

    \indent\rlap{a}b\dotfill \indent\rlap{a}b
    %* 根据网站的说法,参数文本首先在缩进2em的位置打印出来,然后命令后文本向左移动,使得其右端和参数文本的右端重合

    \makebox[0pt][l]{a}b\dotfill \makebox[0pt][l]{a}b
    %# 受帖子的启发,我们可以很清楚地发现,网站的说法实际等同于在命令后内容的左边创建一个宽度为0pt的箱子,并且设置箱子内文本为左对齐

\hrulefill

%? 经过以上讨论,我们发现"\llap{text}"命令和"\rlap{text}"命令其实完全可以转换成更方便我们理解的"\makebox[width][position]{text}"命令,现在可能还剩一个问题,那就是当这两条命令后面存在文本,而前面不存在文本是,应当在命令前使用一条设置距离的命令,否则命令后文本会另起一行打印,并且参数文本会分别在文本左边线的左边和右边打印出来,显然这其中仍然可以看到宽度为0pt的箱子的影子,但是要解答这个问题,看来还需要深入研究这两条命令的底层定义

%* 下面是在此之前还在探索阶段利用"\llap{text}"命令和"\rlap{text}"命令进行的尝试,使用的是和书中一样的汉字的例子,也附在下面,已经作的注释也不删除,可以发现,如果不通过"\makebox[width][position]{text}"命令来辅助理解,对于这两条命令的效果是无法得到完整彻底的解释的,甚至在有些时候还会得到错误的理解

    你看不清这些字\llap{是什么}

    你看不清这些字\llap{是什么,但只是部分看不清}
    %// 如果在"\llap{text}"命令前面有一段文本,参数文本的初始打印位置会从命令前文本的右端向左移动自身长度的距离,最终参数文本的右端和命令前文本的右端重合,如果命令前文本的长度更长,效果就是参数文本被命令前文本完全覆盖,如果参数文本的长度更长,效果就是参数文本的一部分突出到命令前文本的左端之外

    \llap{是什么,而且它在上一段}$\uparrow$\mbox{你看得清这些字}
    %// 如果在"\llap{text}"命令后面有一段文本,参数文本的初始打印位置会从当前所在行对应的页面左边线向左移动自身长度的距离,最终参数文本的右端和页面左边线重合,而命令后文本会另起一段打印出来,正常缩进

    你看不清这些字\llap{是什么}$\leftarrow$\mbox{但是后面这部分除外}
    %// 如果在"\llap{text}"的前后都有文本,参数文本的初始打印位置会从命令前文本的右端向左移动自身长度的距离,最终参数文本的右端和命令前文本的右端重合,命令后文本则会接在命令前文本的右端继续打印

    \indent\llap{是什么,只是语序有些乱}$\leftarrow$\mbox{你看得清这些字}

    \noindent\llap{是什么,只是语序有些乱}$\leftarrow$\mbox{你看得清这些字}

    \hspace{5em}\llap{是什么,只是语序有些乱}$\leftarrow$\mbox{你看得清这些字}

    \hspace{-5em}\llap{是什么,只是语序有些乱}$\leftarrow$\mbox{你看得清这些字}
    %// 如果在"\llap{text}"命令后面有一段文本,在前面有一个"\indent"命令(或者"\noindent"命令,或者任意一段"\hspace{l}"命令设置的长度),参数文本的初始打印位置会从当前所在行对应的页面左边线右边2em(对应"\indent"命令,如果是"\noindent"命令,出发点就在页面左边线,如果是"\hspace{l}"命令,出发点根据设置的距离而定)向左移动自身长度的距离,命令后文本会接着参数文本的右端继续打印

    %// 可以发现,从最终效果来看,只有位于"\llap{text}"命令左边的文本才会对参数文本造成覆盖,而如果"\llap{text}"命令的左边没有文本,一般此时会在"\llap{text}"命令的左边加上一条表示距离的命令,防止命令后文本另起一段打印,此时追求的目的一般是让参数文本在文本区域外打印出来,当然,这种情况下如果进行适当的调整,可以打印出相当于没有使用"\llap{text}"命令的效果:
    \hspace{6em}\llap{这一段你一定}$\leftarrow$\mbox{看得清(使用\textbackslash llap\{text\}\mbox{命令}})

    这一段你一定\mbox{$\leftarrow$\mbox{看得清(正常打印,没有使用任何命令)}}

\hrulefill

    \mbox{你看得清这些字}$\rightarrow$\rlap{是什么}
    %// 如果在"\rlap{text}"命令前面有一段文本,参数文本直接接着命令前文本的右端继续打印
    %! 此处是最容易犯错的地方,如果只看表面效果,会误以为这条命令和直接打印相应文本内容没有什么不同,实际上并非如此

    \rlap{是什么,而且它在上一段}$\uparrow$\mbox{你看得清这些字}
    %// 如果在"\rlap{text}"命令后面有一段文本,参数文本从当前所在行对应的页面左边线开始打印,而后面的文本会另起一段打印出来,正常缩进

    这下\rlap{这些}你也看不清

    这下\rlap{这些,但只是部分看不清}你也看不清
    %// 如果在"\rlap{text}"的前后都有文本,参数文本直接接着命令前文本的右端继续打印,而命令后文本也接着命令前文本的右端继续打印,也就是说,参数文本和命令后文本的左端重合,如果命令后文本更长,效果就是参数文本被命令后文本完全覆盖,如果参数文本更长,效果就是参数文本的一部分突出到命令后文本的右端之外

    \indent\rlap{这些}你也看不清

    \noindent\rlap{这些}你也看不清

    \hspace{5em}\rlap{这些}你也看不清

    \hspace{-5em}\rlap{这些}你也看不清
    %// 如果在"\rlap{text}"命令后面有一段文本,在前面有一个"\indent"命令(或者"\noindent"命令,或者任意一段"\hspace{l}"命令设置的长度),参数文本的会从当前所在行对应的页面左边线右边2em(对应"\indent"命令,如果是"\noindent"命令,打印位置就在页面左边线,如果是"\hspace{l}"命令,打印位置根据设置的距离而定)开始打印,命令后文本会在同一位置继续打印,因此参数文本和命令后文本的左端始终是重合的

    %// 可以发现,从最终效果来看,只有位于"\rlap{text}"命令右边的文本才会对参数文本造成覆盖,一般此时会在"\rlap{text}"命令的左边加上一条表示距离的命令,防止命令后文本另起一段打印,而如果"\rlap{text}"命令的右边没有文本,此时追求的目的一般是让参数文本在文本区域外打印出来,还需要在命令前文本的前面加上一条"\hfill"命令:
    \hfill \mbox{文本区域内}$\rightarrow$\rlap{此处开始已经超出文本区域}

\end{document}