\documentclass{article}
\usepackage{gb4e}
\usepackage{ctex}[scheme=plain]
\usepackage{fontspec}
\newcommand{\ipa}[1]{{\fontspec{Charis SIL}#1}}

%rfr 以下内容改编自:https://zhuanlan.zhihu.com/p/400881951?utm_medium=social&utm_oi=734349454764699648 
%rfr 也可以参考gb4e宏包的官方文档:https://mirrors.ircam.fr/pub/CTAN/macros/latex/contrib/gb4e/gb4e-doc.pdf

\begin{document}

%# 一、列举例句
\begin{exe}
  \ex This is an example sentence.
  \ex C'est un autre exemple.
  \ex これは三番目の例文である。
\end{exe}
% Tip: 要想改变序号格式,则可以通过 \renewcommand{\thexnumi}{\数字格式指令{xnumi}} 可以改变序号格式,默认是 \arabic (1) (2) (3)。在这里把 \数字格式指令 替换为 \roman 则为 (i) (ii) (iii) 等,\Roman 则为 (I) (II) (III),\alph 为 (a) (b) (c), \Alph 为 (A) (B) (C) 等。

%嵌套例句、索引、加星号/问号
\begin{exe}
  \ex\label{first} This is an example sentence. % "\label{}"加在"\ex"后面
  \ex[*]{This an example sentence not is.} % 星号加在中括号里
  \ex C'est trois autres exemples.
    \begin{xlist} % "xlist"第一次嵌套
      \ex これは入れ子例文である。
      \ex[?]{これは入れ子。} % 问号加在中括号里
      \ex[??]{這是更多嵌套例文。}
        \begin{xlist} % "xlist"第二次嵌套
          \ex 你好。
            \begin{xlistI} % "xlist"不再能够嵌套,使用"xlistI"继续嵌套
              \ex 還能嗎?
            \end{xlistI}
          \ex 你不好。
            \begin{xlistA} % "xlist"不再能够嵌套,使用"xlistA"继续嵌套
              \ex 可以。
            \end{xlistA}
        \end{xlist}  
    \end{xlist}
\end{exe}
這個例文(\ref{first}) % 引用相应例句(组)
% Tip: xlist 环境的默认最多嵌套到第三层(exe-xlist-xlist)。第四层往下(没人用得到吧……)必须通过改用 xlistn (arabic)、xlistA (Alph)、xlista (alph)、xlistI (Roman)、xlisti (roman)手动指定序号类型。当然普通层级也可以直接用上面的指定。

\newpage

%# 二、行间标注
\begin{exe}
  \ex{langue} % 大括号中标示语种(但如果遇上汉字内容,比如文言文,可以直接写在这一行,方便下面一行用来注音,如此可以避免使用"glll"增加一行)
  \gll X Y Z\\ % 文字行,注意最后的行结束标记"\\"不能省
  x y z\\ % 标注行,标注行中的大写标注通常用"textsc{}"缩小以确保美观
  \glt traduction % 翻译行,注意翻译行不需要行结束标记,"\glt"也可以替换为"\trans"
\end{exe}
%// 不知为何,将上述带注释内容复制后编译会出现问题,下面是不带注释的内容
% ↑↑↑原因是"\\"后面紧接着"%",结果会报错,因此注释尽量在正文后面空一格输入

\begin{exe}
  \let\eachwordone=\bf
  \let\eachwordtwo=\ipa
  \let\eachwordthree=\textit
  \ex{Middle Chinese}
  \glll 明 月 出 天 山\\ % 文字行
  mjaeŋ ŋjwot tɕʰwit tʰen ʂɛn\\ % 音标行
  bright moon go.out sky mountain\\ % 标注行
  \glt \textit{``From Heaven's Peak the moon rises bright'' (from 许渊冲)} % 翻译行
\end{exe}
% Tip: 使用 \glll 可以变成三行,好加更多信息。使用 \glt 可以提供翻译(可选)。
% "\let\eachwordone=..."这一命令可以将注释部分的第一/二/三行(分别对应\eachwordone、\eachwordtwo、\eachwordthree)文字设定为指定的字体,"="后的命令可以不带论元,如"\bf",也可以带一个论元,如"\textit",也可以是自定义的带一个论元的命令,如此处的"\ipa",这一命令如果放在导言区,则效果应用于全局,如果放在正文环境中,则效果应用于局部

% 嵌套例句
\begin{exe}
  \ex
  \begin{xlist}
      \ex 
      \gll
      A B C\\
      a b c\\
      \glt abcde

      \ex 
      \gll
      A B C\\
      a b c\\
      \glt abcde
  \end{xlist}
\end{exe}

\begin{exe}
  % 继续编号
  \ex{甲:}
  \gll A B C\\
  a b c\\
  \glt ``translation''

  % 继续编号
  \ex{}
  \glll 甲:\\
  A B C\\
  a b c\\
  \glt ``translation''

  % 停止编号
  \sn 乙:
  \gll A B C\\
  a b c\\
  \glt ``translation''

  % 停止编号
  \exi{} 丙: % 内容要放在大括号之外,否则排版会出错
  \gll A B C\\
  a b c\\
  \glt ``translation'' 

  % 继续编号
  \ex{丁:}
  \gll A B C\\
  a b c\\
  \glt ``translation'' 
\end{exe}
% 使用"\sn"或者"\exi{}"(后者的大括号不可省略)替换原来的"\ex",可以使当前所在例句停止编号


\end{document}