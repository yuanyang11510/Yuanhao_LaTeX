\documentclass{article}
\usepackage{gb4e}
\usepackage{ctex}

\begin{document}

%只呈现例句
\begin{exe}
    \ex This is an example sentence.
    \ex C'est un autre exemple.
    \ex これは三番目の例文である。
  \end{exe}
%Tip: 要想改变序号格式,则可以通过 \renewcommand{\thexnumi}{\数字格式指令{xnumi}} 可以改变序号格式,默认是 \arabic (1) (2) (3)。在这里把 \数字格式指令 替换为 \roman 则为 (i) (ii) (iii) 等,\Roman 则为 (I) (II) (III),\alph 为 (a) (b) (c), \Alph 为 (A) (B) (C) 等。

%嵌套例句、索引、加星号/问号
 \begin{exe}
    \ex\label{first} This is an example sentence.%"\label{}"加在"\ex"后面
    \ex[*]{This an example sentence not is.}%星号加在中括号里
    \ex C'est trois autres exemples.
      \begin{xlist}%"xlist"第一次嵌套
        \ex これは入れ子例文である。
        \ex[?]{これは入れ子。}%问号加在中括号里
        \ex[??]{這是更多嵌套例文。}
          \begin{xlist}%"xlist"第二次嵌套
            \ex 你好。
              \begin{xlistI}%"xlist"不再能够嵌套,使用"xlistI"继续嵌套
                \ex 還能嗎?
              \end{xlistI}
            \ex 你不好。
              \begin{xlistA}%"xlist"不再能够嵌套,使用"xlistA"继续嵌套
                \ex 可以。
              \end{xlistA}
          \end{xlist}  
      \end{xlist}
  \end{exe}
  這個例文(\ref{first})%引用相应例句(组)
%Tip: xlist 环境的默认最多嵌套到第三层(exe-xlist-xlist)。第四层往下(没人用得到吧……)必须通过改用 xlistn (arabic)、xlistA (Alph)、xlista (alph)、xlistI (Roman)、xlisti (roman)手动指定序号类型。当然普通层级也可以直接用上面的指定。

%行间标注
%\ex{langue}%大括号中标示语种(但如果遇上汉字内容,比如文言文,可以直接写在这一行,方便下面一行用来注音,如此可以避免使用"glll"增加一行)
%\gll X Y Z\\%文字行,注意最后的行结束标记"\\"不能省
%x y z\\%标注行,标注行中的大写标注通常用"textsc{}"缩小以确保美观
%\glt traduction%翻译行,注意翻译行不需要行结束标记
%\end{exe}

%\begin{exe}
%\ex{langue}
%\glll X Y Z\\%文字行
%a b c\\%音标行
%x y z\\%标注行
%\glt traduction%翻译行
%\end{exe}
%%Tip: 使用 \glll 可以变成三行,好加更多信息。使用 \glt 可以提供翻译(可选)。

%不知为何,将上述带注释内容复制后编译会出现问题,下面是不带注释的内容
\begin{exe}
\ex{langue}
\gll X Y Z\\
x y z\\
\glt traduction
\end{exe}

\begin{exe}
\ex{langue}
\glll X Y Z\\
a b c\\
x y z\\
\glt traduction
\end{exe}

\end{document}